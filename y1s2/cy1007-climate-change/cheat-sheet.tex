% Created 2024-03-29 Fri 14:29
% Intended LaTeX compiler: pdflatex
\documentclass[11pt]{article}
\usepackage[utf8]{inputenc}
\usepackage[T1]{fontenc}
\usepackage{graphicx}
\usepackage{longtable}
\usepackage{wrapfig}
\usepackage{rotating}
\usepackage[normalem]{ulem}
\usepackage{amsmath}
\usepackage{amssymb}
\usepackage{capt-of}
\usepackage{hyperref}
\usepackage{siunitx, array}
\author{Hankertrix}
\date{\today}
\title{Climate Change Cheat Sheet}
\hypersetup{
 pdfauthor={Hankertrix},
 pdftitle={Climate Change Cheat Sheet},
 pdfkeywords={},
 pdfsubject={},
 pdfcreator={Emacs 29.3 (Org mode 9.6.15)}, 
 pdflang={English}}
\begin{document}

\maketitle
\setcounter{tocdepth}{2}
\tableofcontents \clearpage
\section{Definitions}
\label{sec:orgdc6be94}

\subsection{Climate}
\label{sec:org8c69d89}
Climate is defined in terms of the \textbf{average (mean)} of weather elements, such as temperature and precipitation, over a specified period, which is at least \textbf{30 years} according to the World Meteorological Organisation (WMO).

\subsection{Weather}
\label{sec:org040b74f}
Weather is defined as the state of the atmosphere at some place and time, usually expressed in terms of temperature, air pressure, humidity, wind speed and direction, precipitation, and cloudiness.

\subsection{Climate change}
\label{sec:orgdfd5564}
Climate change specifically applies to \textbf{longer-term variations (years and longer)}, in contrast to the shorter fluctuations in weather that last hours, days, weeks, or a few months.

\subsection{Thermal radiation}
\label{sec:orgfa73fc7}
\begin{itemize}
\item Thermal radiation is electromagnetic radiation generated by the thermal motion of particles in matter.
\item All objects emit electromagnetic radiation as long as they have a temperature above absolute zero on the Kelvin (K) scale. The radiation represents a conversion of a body's internal energy into electromagnetic energy.
\end{itemize}

\subsection{Flux}
\label{sec:org4d7db5c}
Flux is the energy per unit area.

\subsection{Albedo}
\label{sec:org6c82593}
The albedo is the fraction of the Sun's energy flux that gets reflected into space.

\newpage

\subsection{Blackbody}
\label{sec:org7712d88}
\begin{itemize}
\item Blackbody is an object that absorbs all radiation falling on it at all wavelengths.
\item Blackbody is an idealised non-reflective body, which does not exist in nature.
\item However, many objects can approximate the behaviour of a blackbody over a limited range of wavelengths or temperatures.
\item The sun emits shortwave radiation due to it being hotter (shorter wavelength means higher energy) while the Earth emits longwave radiation due to it being cooler (longer wavelength means lower energy).
\end{itemize}

\subsection{Differential heating}
\label{sec:orgd2aeba6}
\begin{center}
\includegraphics[width=.9\linewidth]{./images/differential-heating.png}
\end{center}

 \noindent The radiative power absorbed by the Earth decreases with latitude as energy from the sun is constant, but the area increases with latitude.

\subsection{Pressure}
\label{sec:org79c9efb}
Pressure is the \textbf{force exerted per unit are perpendicular to the force}.
\[P = \frac{F}{A}\]

In a fluid, the pressure at a location is the same from all directions.

\subsection{Ideal gas law}
\label{sec:orgc939afa}
\[P = \rho RT\]

Where:
\begin{itemize}
\item \(\rho\) is the density
\item \(T\) is the temperature in \(\unit{K}\)
\item \(R\) is the specific gas constant (\(\qty{287}{J.K^{-1}.kg^{-1}}\) for air)
\end{itemize}

This means that the pressure of air:
\begin{itemize}
\item Increases with increasing density if temperature is kept constant
\item Increases with increasing temperature if density is kept constant.
\end{itemize}

\subsection{Vertical pressure gradient}
\label{sec:org1b8f3aa}
Vertical pressure gradient is the differential of the pressure in the vertical direction, and it is equal to the negative of the density of the fluid multiplied by the gravitational acceleration.
\[\frac{dp}{dz} = - \rho g\]

\subsection{Hydrostatic balance}
\label{sec:org6f66e7e}
\[\frac{\Delta p}{\Delta z} = \frac{dp}{dz} = -\rho g = - \frac{p}{RT}g\]

Hydrostatic balance is the equilibrium that exists between \textbf{the force due to vertical pressure gradient} and the \textbf{weight of a fluid} such that the fluid remains at rest in the vertical direction.
\\[0pt]

This is a \textbf{very good approximation} for the atmosphere and the oceans because vertical motion is generally weak.
\\[0pt]

Since the right-hand-side of the equation is negative, \textbf{pressure must decrease upwards} for the fluid to support its own weight.

\subsection{Meridional}
\label{sec:org2225eb2}
Meridional just means from north to south or vice versa.

\subsection{Trade winds}
\label{sec:org0f120c4}
Trade winds are winds that blow form the east (i.e. easterly winds) throughout the year.

\subsection{Monsoons}
\label{sec:org984824d}
\begin{itemize}
\item Monsoons are winds that blow in one direction for half a year and in the \textbf{opposite direction} in the other half-year.
\item "American monsoons" change directions less dramatically.
\end{itemize}

\subsection{Walker circulation}
\label{sec:orgaa79489}
\begin{itemize}
\item The Walker circulation is the circulation that exists in the equatorial plane of the atmosphere.
\item The Walker circulation blows from east to west.
\item Due to the coupled interaction between the tropical atmosphere and the upper layer of the Pacific Ocean, it undergoes an \textbf{irregular} inter-annual oscillation known the \textbf{El Niño Southern Oscillation (ENSO)}.
\end{itemize}

\subsection{El Niño Southern Oscillation (ENSO)}
\label{sec:org924e39b}
\begin{itemize}
\item El Niño / La Niña refers the phase of the ocean conditions.
\item Southern Oscillation refers to the atmospheric conditions.
\item Southern Oscillation Index (SOI) is the surface pressure at Tahiti minus that at Darwin, Australia. It is used to characterise the phase of ENSO.
\end{itemize}

\subsubsection{El Niño}
\label{sec:orgf984ee3}
\begin{itemize}
\item El Niño is basically a reversal or relaxation of the walker circulation.
\item Trade winds weaken or even change direction and blow the other way, and the warm water piled up near Australia sloshes to the east.
\item During El Niño, Australia and Asia's temperature will \textbf{decrease}, and South America's temperature will \textbf{increase}.
\end{itemize}

\subsubsection{La Nina}
\label{sec:orgfd9b9f0}
\begin{itemize}
\item During La Niña, Australia and Asia's temperature will \textbf{increase}, and South America's temperature will \textbf{decrease}.
\end{itemize}

\subsection{Cyclones}
\label{sec:org98b39b3}
Cyclones are low-pressure systems at the surface. Cyclones have \textbf{surface} winds that spiral \textbf{inward} in the \textbf{anticlockwise} direction in the \textbf{Northern} Hemisphere. The rotation of the cyclone is \textbf{opposite} in the \textbf{Southern} Hemisphere, which means the surface winds spiral \textbf{outward} in the \textbf{clockwise} direction in the \textbf{Southern} Hemisphere.

\subsubsection{Extratropical cyclone}
\label{sec:org5cb451a}
\begin{itemize}
\item Extratropical cyclones are cyclones that occur in areas outside tropical and equatorial areas.
\item Mid-latitude weather is dominated by extratropical cyclones.
\end{itemize}

\subsection{Anticyclones}
\label{sec:org1906adb}
Anticyclones are high-pressure systems at the surface. They have \textbf{surface} winds that spiral \textbf{outward} in the \textbf{clockwise} direction in the \textbf{Northern} Hemisphere. The rotation of the anticyclone is \textbf{opposite} in the \textbf{Southern} Hemisphere, which means the surface winds spiral \textbf{inward} in the \textbf{anticlockwise} direction in the \textbf{Southern} Hemisphere.

\subsection{Tropical cyclones (TC)}
\label{sec:orga4bb858}
\begin{itemize}
\item They are circulating storms that have a typical diameter of about \(\mathbf{\qty{500}{km} - \qty{2000}{km}}\) (for \(\text{winds} > \qty{15}{m.s^{-1}}\)).
\item They have the \textbf{warmest} air and the \textbf{lowest pressure} at the centre and last a few days.
\item The surface flow is \textbf{anticlockwise} in the \textbf{Northern} Hemisphere and clockwise in the \textbf{Southern} Hemisphere.
\item Tropical cyclones have become more intense, and more destructive tropical cyclones are forming more often.
\end{itemize}

\newpage

\subsection{Madden-Julian Oscillations (MJO)}
\label{sec:org12de049}
\begin{itemize}
\item Madden-Julian Oscillations are part of a planetary-scale mode of tropical variability, in which convection develops over the western tropical Indian Ocean south of the equator, propagates slowly eastward across the Indian Ocean and Maritime Continent
\item It is a coupled ocean-atmosphere system with a 30 - 60 day oscillation in the tropical atmosphere, with super-clusters of cloud or rain \textbf{moving eastward} from the East African coast across the Indian Ocean, through Southeast Asia, and into the Pacific Ocean.
\end{itemize}

\subsection{Rossby waves}
\label{sec:org96770ca}
Rossby waves are \textbf{westward} moving \textbf{equatorial} waves of convection along the equator for 3 - 10 days.

\subsection{Kelvin waves}
\label{sec:orgf165381}
Kelvin waves are \textbf{eastward} moving \textbf{equatorial} waves of convection along the equator for 3 - 10 days.

\subsection{Cold surge}
\label{sec:org773d814}
Cold surge events typically occur during the \textbf{northeast winter monsoon season}. Bursts of cool dense air from the Asian continent penetrate into the South China Sea and persist for a few days.

\subsection{Borneo vortex}
\label{sec:org49619af}
A Borneo vortex typically appears off the northwestern coast of northern Borneo. When a Borneo vortex encounters a cold surge, they can spin up into an \textbf{anticlockwise vortex} near Borneo island and bring lots of rain into the region.

\newpage

\subsection{Squall}
\label{sec:orgc6f466a}
Squall is just a long line of clouds.

\subsection{Sumatra squalls}
\label{sec:org6a98865}
\begin{itemize}
\item Sumatra squalls are initiated by convergent effects at night over the Sumatra west coast or the Malacca Straits, and then move east towards Singapore and the Malaysian Peninsula during the early morning, and later dissipate in the South China Sea.
\item They typically appear during \textbf{the southwest summer monsoons season}, bringing in wind gusts and thunderstorms.
\end{itemize}

\subsection{Bathymetry}
\label{sec:org9b79fb8}
Bathymetry refers to the topography of the sea floor.

\subsection{Thermohaline conveyor belt}
\label{sec:org0387dc5}
\begin{center}
\includegraphics[width=.9\linewidth]{./images/thermohaline-conveyor-belt.png}
\end{center}

\begin{enumerate}
\item Deep water formation
\item Spreading of deep water
\item Upwelling
\item Near-surface currents
\end{enumerate}

\subsection{Ocean eddies}
\label{sec:org7101057}
\begin{itemize}
\item Ocean eddies are created by \textbf{surface winds} on weather timescales, by \textbf{bathymetry (sea-floor topography)} and by instabilities in the currents.
\item Ocean eddies are relatively small, contained pockets of moving water that break off from the main body of a current and travel independently of their parent. They can form in almost any part of a current.
\item Ocean eddies, in a sense, are the weather of the ocean. They do not only influence heat uptake or heat transport in the oceans, but also the distribution of nutrients or the ability to absorb carbon from the atmosphere. This makes them important for global ocean circulation.
\end{itemize}

\subsection{Waves}
\label{sec:org8496b4b}
\begin{itemize}
\item Surface waves are created by \textbf{friction with surface winds}.
\item They can grow very large as they propagate into shallow seas.
\item Tsunamis are created by underwater \textbf{earthquakes}, \textbf{landslides}, \textbf{volcanic eruptions}.
\end{itemize}

\subsection{Vapour pressure (\(e\))}
\label{sec:org9bb41e3}
Vapour pressure (\(e\)) is the partial pressure exerted by the water vapour in moist air. It is a measure of the amount of water vapour in the air.

\newpage

\subsection{Saturated vapour pressure (\(e_s\))}
\label{sec:org3878959}
Saturated vapour pressure with respect to water or ice is the vapour pressure attained when the water vapour is in equilibrium with liquid water or ice.

\subsubsection{Bolton (1980) empirical formula}
\label{sec:orgd17aa6e}
\[e_s \approx 6.112e^{\frac{17.67 T_C}{T_C + 243.5}}\]

Where:
\begin{itemize}
\item \(T_C\) is the temperature in degrees Celsius
\end{itemize}

Saturated vapour pressure increases exponentially with \(T\), which is the \emph{Clausius-Clapeyron} relationship.
\\[0pt]

When \(e > e_s\), condensation occurs.

\subsection{Relative humidity (\(H\))}
\label{sec:org2dcc7d0}
Relative humidity is the ratio of the observed vapour pressure to the saturated vapour pressure with respect to water or ice at the \textbf{same temperature and pressure}. The relative humidity is expressed as a percentage.
\[H = \frac{e}{e_s} \times 100\% \]

\subsection{Specific humidity (\(q\))}
\label{sec:orgf334e12}
Specific humidity is the mass of water vapour per unit mass of moist air.

\[q = \frac{\rho_v}{\rho_m}\]

Where:
\begin{itemize}
\item \(\rho_v\) is the density of water vapour
\item \(\rho_m\) is the density of moist air
\end{itemize}

\subsection{Adiabatic}
\label{sec:orgb806a81}
Adiabatic means that there is no net input of heat.

\subsection{Adiabatic lapse rate}
\label{sec:orgadf773a}
\begin{itemize}
\item The rate at which temperature decreases in a rising, expanding air parcel.
\item Dry adiabatic lapse rate, i.e., without condensation produced, is a constant at \(\qty{9.8}{\degreeCelsius.km^{-1}}\).
\item However, if moist air saturates and condensation occurs, saturated adiabatic lapse rate can be significantly lower, going down to \(\qty{4}{\degreeCelsius.km^{-1}}\).
\end{itemize}

\subsection{Paleoclimate}
\label{sec:orgbf9730f}
Paleoclimate refers to the climate of the Earth at a specified point in geologic time.

\subsection{Paleoclimate archives}
\label{sec:orgb9ae151}
Paleoclimate archives consist of geologic (e.g. sediment cores) and biologic (e.g. tree rings) materials that \textbf{preserve evidence of past changes in climate}.

\subsubsection{Examples}
\label{sec:org4425e54}
\begin{itemize}
\item Corals
\item Marine and lake sediments
\item Sediment cores
\item Tree rings
\item Ice cores
\item Speleotherms (cave carbonate deposits)
\end{itemize}

\subsection{Paleoclimate proxies}
\label{sec:org43d88b3}
Paleoclimate proxies are \textbf{preserved physical characteristics} of the \textbf{past} that \textbf{stand in} for direct meteorological measurements and enable scientists to reconstruct the climatic conditions over a longer fraction of the Earth's history.

\subsubsection{Examples}
\label{sec:org4d46580}
\begin{itemize}
\item Physical properties
\item Chemical composition
\end{itemize}

\subsubsection{Using oxygen as a climate proxy}
\label{sec:org2a9e822}
\begin{itemize}
\item In 1950, Harold Urey determined that the abundance of \(^{18}O\) relative to \(^{16}O\) in carbonates could be used as a way of determining temperature.
\item Foraminifera use oxygen in the seawater to make their shells.
\item If the water molecules in the ocean are relatively richer in \(^{18}O\) when they form their shells (like during an ice age), the foraminifera's shells will also have a relatively greater concentration of \(^{18}O\). Hence, their tiny shells are chemical records of the waxing and waning of Earth's great ice sheets.
\end{itemize}

\subsubsection{Using ice cores as a climate proxy}
\label{sec:orgcc84e8e}
\begin{itemize}
\item The interglacial period has less land ice. Hence, there is more \(^{18}O\) in ice, and less \(^{18}O\) in the ocean.
\item The glacial period means there is more land ice. Hence, there is less \(^{18}O\) in ice and more \(^{18}O\) in the ocean.
\item Temperature is positively correlated with \(CO_2\) and \(CH_4\), and negatively correlated with ice volume. But a change in global-mean surface temperature of about \(4 - 5 \ \unit{\degreeCelsius}\) between glacial periods and interglacial periods is a massive climate shift.
\item Glacial-interglacial cycles have a sawtooth shape and there is an abrupt warming towards the end of the glacial periods.
\item The end of glacial periods are called terminations.
\end{itemize}

\subsection{Circumpolar}
\label{sec:orgd522938}
Circumpolar means something is located or found in one of the polar regions, so either in the Arctic or in Antarctica.

\subsection{Anthropocene}
\label{sec:orgd8483de}
\begin{itemize}
\item Anthropocene refers to the age of humans.
\item The proposed beginning of the Anthropocene is 1950.
\end{itemize}

\subsection{Global warming}
\label{sec:org2378a04}
\begin{itemize}
\item Global warming is the phenomenon of increasing average air temperatures near the Earth surface (\textbf{surface warming}) over the past \textbf{100 - 200} years.
\item In contrast, the stratosphere is experiencing cooling (\textbf{stratospheric cooling}).
\item From 1986 to 2022, temperatures declined in the higher levels of Earth's atmosphere, while increasing in the layers of the atmosphere closest to the Earth's surface.
\item The mesosphere and lower thermosphere are also experiencing cooling.
\item \textbf{Stratospheric cooling is faster than surface warming}.
\end{itemize}

\subsection{Heat capacity}
\label{sec:org20eef40}
Heat capacity is the amount of heat required to raise the temperature of the object by exactly \(\qty{1}{\degreeCelsius}\) (or \(\qty{1}{K}\)).

\begin{itemize}
\item Specific heat capacity (one gram)
\item Molar heat capacity (one mole)
\end{itemize}

\[\text{Heat capacity} = \text{Specific heat capacity} \times \text{mass}\]

\begin{center}
\begin{tabular}{|m{8em}|m{10em}|m{10em}|m{5em}|}
\hline
Substance & Specific heat capacity at \(\qty{25}{\degreeCelsius}\) (\(\unit{J.g^{-1}.K^{-1}}\)) & Earth system & Mass (\(\unit{kg}\))\\[0pt]
\hline
Water (liquid) & 4.18 & Oceans (covers 70\% of the Earth's surface) & \(1.4 \times 10^{21}\)\\[0pt]
\hline
Air (typical room conditions) & 1.012 & Atmosphere & \(5.15 \times 10^{18}\)\\[0pt]
\hline
Soil or sand & Roughly 0.8 & Land (Solid Earth) & \(6.0 \times 10^{24}\)\\[0pt]
\hline
\end{tabular}
\end{center}

\newpage

\subsection{Orbital cycles}
\label{sec:org14b5f87}
Three periodic motions in Earth's orbit, known as Milankovitch cycles contribute a predictable amount of variation to Earth's climate over timeframes of tens of thousands to hundreds of thousands of years.

\subsubsection{Changes in eccentricity (orbit shape)}
\label{sec:org405b9a5}
100,000-year cycles

\subsubsection{Axial precession (wobble)}
\label{sec:org8ed10dc}
26,000-year cycles

\subsubsection{Changes in obliquity (tilt)}
\label{sec:org2b280a9}
41,000-year cycles

\subsection{Anthropogenic CO\textsubscript{2} multiplier (ACM)}
\label{sec:orgd77f144}
The anthropogenic CO\textsubscript{2} multiplier (ACM) is the ratio of annual anthropogenic CO\textsubscript{2} to maximum preferred estimate for annual volcanic CO\textsubscript{2}, which is an index of anthropogenic CO\textsubscript{2}'s dominance over volcanic CO\textsubscript{2} emissions.

\subsection{Carbon cycle}
\label{sec:org8752ebf}
\begin{itemize}
\item The carbon cycle is a whole system of processes that move the element carbon in various forms through the Earth's biosphere (living matter), atmosphere (air), hydrosphere (water), cryosphere (frozen ground), and geosphere (land).
\item As only a tiny number of atoms reach the Earth from space, the Earth is a closed system.
\item \textbf{Earth does not gain or lose carbon. But carbon does move constantly from place to place.}
\end{itemize}

\newpage

\subsubsection{Importance of the carbon cycle}
\label{sec:org0f8178a}
\begin{itemize}
\item Carbon is the 4th most abundant element in the universe.
\item Carbon is the foundation of all life on Earth, required to form complex molecules like proteins and DNA.
\item Carbon is a key ingredient in the food that sustains us (carbohydrates).
\item Carbon provides a major source of the energy consumed by human civilisation (fossil fuels).
\item Carbon helps to regulate the Earth's temperature (CO\textsubscript{2}).
\end{itemize}

\subsubsection{Forms of carbon in the non-living environments}
\label{sec:org3922368}
\begin{itemize}
\item Carbon dioxide (CO\textsubscript{2}) in the air
\item Carbon dioxide dissolved in water to form the bicarbonate ion \(HCO_{3}^{-}\)
\item Carbonate rocks such as limestone (\(CaCO_{3}\))
\item Dead organic matter in the soil such as humus
\item Fossil fuels from dead organic matter (coal, oil natural gas)
\end{itemize}

\subsubsection{Main reservoirs of carbon}
\label{sec:org86c391f}
On Earth, most carbon is stored in rocks and sediments, while the rest is in the ocean, atmosphere, and in living organisms.
\begin{center}
\begin{tabular}{l|r}
\textbf{Reservoir} & Amount of carbon (GT or 10\textsuperscript{15}g)\\[0pt]
\hline
Mantle & Huge amount (exact amount unknown)\\[0pt]
Sedimentary rocks & 1,000,000\\[0pt]
Deep oceans & 38,000\\[0pt]
Soil & 1580\\[0pt]
Surface oceans & 970\\[0pt]
Atmosphere & 750\\[0pt]
Land biota & 610\\[0pt]
Ocean biota & 3\\[0pt]
\end{tabular}
\end{center}

\subsubsection{Exchanges between the main carbon reservoirs}
\label{sec:org5676843}
\begin{center}
\includegraphics[width=.9\linewidth]{./images/carbon-reservoir-exchanges.png}
\end{center}

\newpage

\subsubsection{Fast carbon cycle}
\label{sec:org6ee1f51}
\begin{center}
\includegraphics[scale=0.55]{./images/fast-carbon-cycle-diagram.png}
\end{center}

\begin{center}
\includegraphics[width=.9\linewidth]{./images/fast-carbon-cycle-picture.jpg}
\end{center}

\subsubsection{Slow carbon cycle}
\label{sec:orgd0a1500}
\begin{center}
\includegraphics[scale=0.6]{./images/slow-carbon-cycle-diagram.png}
\end{center}

\begin{center}
\includegraphics[width=.9\linewidth]{./images/slow-carbon-cycle-picture.jpg}
\end{center}

\subsubsection{Comparison between the carbon cycles}
\label{sec:orge6be570}
Human emissions due to fossil fuel burning as well and land-use changes such as forest cutting, soil erosion, forest burning, and soil disruption due to ploughing have altered the carbon cycle.
\begin{center}
\begin{tabular}{l|l|l}
\textbf{Carbon cycle} & \textbf{Period} & \textbf{Carbon moved or emitted per year}\\[0pt]
\hline
Slow cycle & 100 - 200 million years & 10 to 100 million metric tons\\[0pt]
Fast cycle & Years to decades & 1,000 to 100,000 million metric tons\\[0pt]
Human emission & 100 - 200 years & 10,000 million metric tons\\[0pt]
\end{tabular}
\end{center}

\subsubsection{Marine carbon cycle}
\label{sec:org2fc6517}
\begin{center}
\includegraphics[width=.9\linewidth]{./images/marine-carbon-cycle.png}
\end{center}

\newpage

\subsection{Radiative forcing}
\label{sec:org3e7d647}
\begin{itemize}
\item Radiative forcing is the change in average net radiation at the top of the troposphere which occurs because of a change in the concentration of a greenhouse gas or because of some other change in the overall climate system.
\item A \textbf{positive} radiative forcing tends to \textbf{warm} the surface on average and a \textbf{negative} radiative forcing tends to \textbf{cool} the surface on average.
\item It is also a measure of how the energy balance of the Earth-atmosphere system is influenced, and is the change in the balance between incoming solar radiation and outgoing IR radiation within the Earth's atmosphere.
\item Forcing indicates that Earth's radiative balance is pushed away from its normal state.
\end{itemize}

\subsubsection{Representative concentration pathways (RCPs)}
\label{sec:orgce819be}
\begin{center}
\begin{tabular}{|c|m{17em}|c|}
\hline
\textbf{RCP} & \textbf{Timeline of peak radiative forcing} & \textbf{Peak CO\textsubscript{2} concentration}\\[0pt]
\hline
8.5 & More than 8.5 W m\textsuperscript{-2} by 2100 and constant after 2250 & 1370 ppm\\[0pt]
\hline
6.0 & Peaks at around 6.0 W m\textsuperscript{-2} before 2100, and then declines. It is constant after 2150 & 850 ppm\\[0pt]
\hline
4.5 & Peaks at around 4.5 W m\textsuperscript{-2} before 2100 and then declines. It is constant after 2150 & 650 ppm\\[0pt]
\hline
2.6 & Peaks at around 3 W m\textsuperscript{-2} before 2100 and then declines & 490 ppm\\[0pt]
\hline
\end{tabular}
\end{center}

\subsection{Carbon sinks}
\label{sec:org6c62add}
Carbon sinks refer to places where carbon is stored away from the atmosphere.

\subsection{Carbon budget}
\label{sec:orgd0074e9}
Carbon budget refers the total net amount of CO\textsubscript{2} that can still be emitted by human activities while limiting global warming to a specified level.

\subsection{Earth Observatory of Singapore}
\label{sec:orgd5d58dc}
The Earth Observatory of Singapore conducts fundamental research on earthquakes, volcanic eruptions, tsunamis and climate change in and around Southease Asia, toward safer and more sustainable societies.

\subsection{King tide}
\label{sec:org11006b6}
When the Earth, Sun, and Moon are aligned such that the tides at the highest levels. This happens twice a year.

\subsection{Pro-glacial forebulge}
\label{sec:org6eab42f}
A pro-glacial forebulge is formed due to the weight of an ice sheet pressing down on the earth, lifting the surrounding land around the ice sheet up.

\subsection{Hazard}
\label{sec:org9cec401}
The potential occurrence of a natural or human-induced \textbf{physical event or trend} that may cause loss of life, injury, or other health impacts, as well as damage and loss to property, infrastructure, livelihoods, service provision, ecosystems and environmental resources.

\newpage

\subsection{Climate hazard}
\label{sec:orgaa7825e}
Climate-related physical events or trends that have the potential to cause damage and loss. Climate hazards can be climatological, meteorological, or hydrological.

\subsubsection{Physical events}
\label{sec:orgf1ac4f5}
\begin{itemize}
\item Flood
\item Tornado
\item Tropical cyclone
\item Convective storm
\item Drought
\item Haze
\item Wildfire
\item Heat wave
\item Extreme temperature
\end{itemize}

\subsubsection{Trends}
\label{sec:org6da7ed7}
\begin{itemize}
\item Global warming
\item Sea level rise
\item Drying trend
\item Ocean acidification
\end{itemize}

\subsection{Climate disaster}
\label{sec:org4b47290}
\begin{itemize}
\item Events caused by climate hazards, resulting in severe damage and loss.
\item \textbf{Not all hazards} result in disasters!
\end{itemize}

\subsection{Extreme event}
\label{sec:org0421976}

\subsubsection{General definition}
\label{sec:orge01c8c0}
\begin{itemize}
\item Extreme \textbf{weather} event: an event that is \textbf{rare} at a particular \textbf{place and time} of the year.
\item Extreme \textbf{climate} event: a pattern of extreme weather that persists from some time, such as a season.
\item For simplicity, both extreme weather events and extreme climate events are referred to collectively as \textbf{"climate extremes"}.
\end{itemize}

\subsubsection{Statistical definition}
\label{sec:orgb3f816d}
The occurrence of a value of a weather or climate variable above or below a threshold value near the upper or lower ends of the range of observed values of the variable.
\begin{itemize}
\item Relative (e.g. 90th percentile or 10th percentile) thresholds
\item Absolute (e.g. \(\qty{35}{\degreeCelsius} for a hot day\)) thresholds
\end{itemize}

\subsubsection{Changes in extremes}
\label{sec:org2550dce}
Changes in extremes can be examined from two perspectives:
\begin{enumerate}
\item Changes in the frequency for a given magnitude of extremes.
\item Changes in the magnitude for a particular return period (frequency).
\end{enumerate}

\subsection{Climate risk framework}
\label{sec:orgf7fc569}

\subsubsection{Exposure}
\label{sec:org503290c}
The \textbf{presence} of people, livelihoods, species or ecosystems, environmental functions, services, and resources, infrastructure or economic, social, or culrutal assets in places and settings that could be adversely affected.

\subsubsection{Vulnerability}
\label{sec:org586470c}
The \textbf{propensity or predisposition} to be adversely affected. Vulnerability encompasses a variety of concepts and elements including sensitivity or susceptibility to harm and lack of capacity to cope and adapt.

\subsubsection{Risk}
\label{sec:org6bfcc16}
The potential for \textbf{adverse} consequences for \textbf{human or ecological systems}, recognising the diversity of values and objectives associated with such systems.
\begin{itemize}
\item Risks can arise from the dynamic interactions among climate-related hazards, the exposure and vulnerability of affected human and ecological systems.
\item Risk can arise from potential impacts of climate change as well as human responses to climate change.
\end{itemize}

\subsubsection{Impacts}
\label{sec:org3235520}
The consequences of realised risks on \textbf{natural and human systems}, where risks result from the interaction of climate-related hazards, exposure, and vulnerability.
\begin{itemize}
\item Impacts generally refer to effects on lives, livelihoods, health and well-being, ecosystems and species, economic, social and cultural assets, services (including ecosystem services), and infrastructure.
\end{itemize}

\subsection{Climate impact-driver (CID)}
\label{sec:org96d5b67}
A physical climate system condition that directly affects society or ecosystems. Depending on system tolerance, CIDs and their changes can be \textbf{detrimental, beneficial, neutral, or a mixture} of each across interacting system elements and regions. These conditions include:
\begin{enumerate}
\item \textbf{Mean}: A long-term average condition (such as the average winter temperatures that affect indoor heating requirements)
\item \textbf{Event}: A common event (such as a frost that kills off warm-season plants)
\item \textbf{Extreme}: An extreme event (such as a one-in100-year flood that destroys homes)
\end{enumerate}

A CID or its change caused by climate change is not universally hazardous or beneficial, but we refer to it as a "hazard" when experts determine it is detrimental to a specific system. CIDs includes climate hazards.

\subsection{Climatic threshold}
\label{sec:org37c4dfd}
A level beyond which there are either gradual changes in system behaviour or abrupt, non-linear and potentially irreversible impacts.

\subsubsection{Natural thresholds}
\label{sec:org3bb298f}
For example, the growth of a crop.

\subsubsection{Structural thresholds}
\label{sec:org5087c2c}
For example, the height at which a building is built at.

\subsection{Thermodynamic changes}
\label{sec:org81e4bee}
The local exchanges in heat, moisture, and other related quantities.

\subsection{Dynamic changes}
\label{sec:orgba2d9ca}
Changes associated with atmospheric and oceanic motions.

\subsection{Global climate model (GCM)}
\label{sec:orgf1c149e}
\begin{itemize}
\item A global climate model uses hundreds of mathematical equations to describe processes that happen on our planet, processes like wind, ocean currents, and plant growth. Maths and Physics are also used to describe how Earth processes are related to each other.
\item A global climate model typically contains enough computer code to fill 18,000 pages of printed text. It takes hundreds of scientists many years to build and improve.
\end{itemize}

\subsubsection{Importance}
\label{sec:org3b95c41}
\begin{itemize}
\item We do not have observations from every portion of Earth.
\item It allows for better understanding of the present climate system and its sensitivity to external perturbations.
\item Future climate cannot be extrapolated from the past, as future concentration of greenhouse gases is unknown.
\item Climate models help us attribute past climate changes to quantify the contribution from natural forcing and human-induced forcing.
\end{itemize}

\subsection{Courate-Friedrichs-Lewy (CFL) condition}
\label{sec:org8a849de}
\begin{itemize}
\item This condition is needed for numerical simulations.
\item The computational time step is less than the wave travel time to adjacent grid points.
\end{itemize}

\subsection{Coupled model intercomparison project (CMIP)}
\label{sec:org3236562}
\begin{itemize}
\item WCRP working group on climate modelling (WGCM) fosters the development and review of coupled climate models.
\item WGCM formed (1995) CMIP to better understand past, present, and future climate changes in response to radiative forcing in a multimodel context.
\item CMIP phase 3 (CMIP3) was in 2007.
\item CMIP phase 5 (CMIP3) was in 2009.
\item CMIP phase 6 (CMIP3) was in 2016.
\end{itemize}

\subsection{Socioeconomic pathways (SSPs)}
\label{sec:org38612fd}
\begin{itemize}
\item SSPs are scenarios of projected socioeconomic global changes up to 2100
\item They are used to derive greenhouse gas emissions scenarios with different climate policies, like population, economic growth, and urbanisation, that could shape our societies.
\end{itemize}

\subsection{Uncertainties in climate projections}
\label{sec:orgdb9564c}
\begin{itemize}
\item Uncertainty in climate projections refers to a value or relationship that is unknown. Uncertainty can be represented by quantitative measure.
\item In other words, uncertainty is any departure from complete deterministic knowledge of the relevant system.
\item As time progresses, uncertainty increases and the relative role of internal variability decreases.
\item As time progresses, human sources of uncertainties, like the differences among RCPs, increases.
\end{itemize}

\subsubsection{Internal variability}
\label{sec:orgb961aea}
Very similar initial conditions or forcing leads to different results for a single climate model.

\subsubsection{Model uncertainty}
\label{sec:org65ff72f}
Different models yield different results for the same emission scenario.

\subsubsection{Scenario uncertainty}
\label{sec:org19eb9eb}
Spread of model solutions created using different RCP scenarios, related to out lack of knowledge in how future anthropogenic emissions will evolve.

\subsubsection{Dealing with uncertainties}
\label{sec:org336df89}
Understanding and incorporating uncertainties in climate projections is essential for robust decision-making and mitigation of climate change risks. The ways to deal with uncertainties include:
\begin{itemize}
\item Incorporating multiple scenarios.
\item Incorporating multimodel or large-ensemble simulation results
\item Incorporating multiple sources (model, paleo-climate data, observations, expert judgment, etc.)
\item Enhanced effort on climate model development and reduction in model biases.
\end{itemize}

\newpage

\section{Radiative balance model}
\label{sec:orga9f7cf9}
\begin{itemize}
\item The Sun's energy flux \(F_{s}\) comes through a disk of radius \(a\), which is the Earth's radius.
\item A fraction \(A\) (albedo) of the Sun's energy flux gets reflected into space.
\item The remaining fraction \(1-A\) is absorbed by the Earth. The Earth re-emits the radiation into space uniformly in all directions.
\item Assuming the Earth is a blackbody, the radiative flux emitted by the Earth can be calculated using the Stefan-Boltzmann law.
\[F_B = \sigma T^4\]

Where:
\begin{itemize}
\item T is the absolute temperature of the body and \(\sigma\) is the Stefan-Boltzmann constant \(5.670 \times 10^{-8} \unit{W.m^{-2}.K^{-4}}\)
\end{itemize}

\item Radiative balance requires that the energy received from the Sun equals the energy emitted by the Earth:
\[F_s (1-A) \pi a^2 = F_B (4 \pi a^2)\]
\[F_B = \frac{1}{4} F_s(1-A)\]

\item Plug in the Stefan-Boltzmann law:
\[\sigma T^4 = \frac{1}{4} F_s (1-A)\]
\[T = \left[ \frac{F_s(1-A)}{4 \sigma} \right] ^{\frac{1}{4}}\]
\end{itemize}

\subsection{With atmosphere}
\label{sec:org692b341}

\begin{center}
\includegraphics[scale=0.6]{./images/radiative-balance-model-with-atmosphere.png}
\end{center}

\begin{itemize}
\item Assume the Earth atmosphere transmits fraction \(S\) of the incident solar flux \(F_0\) and a fraction \(L\) of the incident terrestrial flux \(F_g\) at the surface (ground or sea).
\item Having absorbed this radiation, the atmosphere re-emits with a flux of \(F_a\) back to the Earth's surface and to space.
\item From above, we know that the average incident solar flux that is absorbed by the Earth is less than what is scattered back to space, specifically:
\[F_0 = \frac{1}{4} F_s (1-A)\]

\item At the top of the atmosphere, radiative balance requires:
\[F_0 = F_a + LF_g \Rightarrow F_a = F_0 - LF_g\]

\item At the bottom of the atmosphere, radiative balance requires:
\[F_a + SF_0 \Rightarrow F_a = F_g - SF_0\]

\item Hence:
\[F_0 - LF_g = F_g - SF_0\]
\[F_g = F_0 \frac{1+ S}{1 + L}\]
\[T_g = \left[\frac{F_s(1 - A)}{4 \sigma} \frac{1+S}{1+L} \right]^{\frac{1}{4}} \quad \left(\because \ F_0 = \frac{1}{4} F_s (1 - A) \right)\]
\end{itemize}

\subsection{Summary}
\label{sec:orgaa2c8c8}

\subsubsection{Without atmosphere}
\label{sec:org5d7e1d3}
\[T = \left[\frac{F_s(1-A)}{4 \sigma} \right]^{\frac{1}{4}}\]

\subsubsection{With atmosphere}
\label{sec:org40cc3a6}
\[T = \left[\frac{F_s(1-A)}{4 \sigma} \frac{1+S}{1+L} \right]\]


\section{Structure of the atmosphere}
\label{sec:org0caff72}
From the lowest layer to the highest layer:
\begin{center}
\begin{tabular}{l|l|l}
Name & Height above the ground (km) & Temperature range (\(\unit{\degreeCelsius}\))\\[0pt]
\hline
Troposphere & 9 to 16 & 30 to -75\\[0pt]
Stratosphere & 15.5 to 50 & -75 to 0\\[0pt]
Mesosphere & 50 to 85 & 0 to -90\\[0pt]
Thermosphere & 85 to 600 & -90 to 1500+\\[0pt]
Exosphere & 600 to 100,000 & ?\\[0pt]
\end{tabular}
\end{center}

\newpage

\section{Atmospheric composition}
\label{sec:orgeae384e}
\begin{center}
\begin{tabular}{l|r}
Gas & Percentage of atmosphere (\%)\\[0pt]
\hline
Nitrogen (N\textsubscript{2}) & 78.084\\[0pt]
Oxygen (N\textsubscript{2}) & 20.946\\[0pt]
Argon (Ar) & 0.9340\\[0pt]
Carbon Dioxide (CO\textsubscript{2}) & 0.0417\\[0pt]
\end{tabular}
\end{center}

\subsection{Trace gases}
\label{sec:org075d333}
\begin{center}
\begin{tabular}{l|r}
Gas & In ppm\\[0pt]
\hline
Neon (Ne) & 18.18\\[0pt]
Helium (He) & 5.24\\[0pt]
Methane (CH\textsubscript{4}) & 1.87\\[0pt]
Krypton (Kr) & 1.14\\[0pt]
Hydrogen (H\textsubscript{2}) & 0.55\\[0pt]
Nitrous Oxide (N\textsubscript{2}O) & 0.50\\[0pt]
Xenon (Xe) & 0.09\\[0pt]
Ozone (O\textsubscript{3}) & 0.07\\[0pt]
Nitrogen Dioxide (NO\textsubscript{2}) & 0.02\\[0pt]
Iodine (I) & 0.01\\[0pt]
Water Vapour (H\textsubscript{2}O) & 0 - 30,000 (0\% - 3\% of the atmosphere)\\[0pt]
\end{tabular}
\end{center}

The greenhouse gases are methane (CH\textsubscript{4}), Nitrous Oxide (N\textsubscript{2}O), Ozone (O\textsubscript{3}), Nitrogen Dioxide (NO\textsubscript{2}) and water vapour (H\textsubscript{2}O).


\section{Global mean energy budget of the Earth}
\label{sec:org0387535}
\begin{center}
\includegraphics[width=.9\linewidth]{./images/global-mean-energy-budget.png}
\end{center}

\section{Global atmospheric conditions}
\label{sec:orgbebf0f9}
\begin{center}
\includegraphics[width=.9\linewidth]{./images/global-atmospheric-circulations.png}
\end{center}

\section{The climate system}
\label{sec:org9184c3c}
\begin{itemize}
\item The atmosphere - the air surrounding the Earth; the most variable part of the climate system.
\item The hydrosphere - the part of the Earth's surface where water is in liquid form, including oceans, lakes and rivers, etc.
\item The cryosphere - the part of the Earth's surface where water is in solid form, including glaciers, ice sheets and frozen ground, etc.
\item The geosphere - the solid parts of the Earth (land).
\item The biosphere - all living things on the Earth (ecosystem)
\item The anthroposphere (people)
\end{itemize}

\newpage

\section{Atmospheric mass distribution}
\label{sec:org4d0676f}
\[\frac{dp}{dz} = - \rho g = - \frac{p}{RT}g \qquad \frac{dp}{p} = - \frac{g}{RT} \, dz\]
\begin{itemize}
\item Pressure in the atmosphere falls roughly \textbf{exponentially} with height.
\item From the ideal gas law, the atmospheric density also falls roughly \textbf{exponentially} with height (when the temperature is roughly constant).
\item \textbf{Pressure} and \textbf{density} falls roughly by a factor of \(e \approx 2.718\) for every ascent of \(h\) in height in the atmosphere.
\[H = \frac{RT}{g}\]

\item Assuming that \(T = \qty{250}{K}\) on average,
\[H = \frac{\left(\qty{287}{J.K^{-1}.kg^{-1}} \right) \left(\qty{250}{K} \right)}{\qty{9.8}{m.s^{-2}}} = \qty{7300}{m} = \qty{7.3}{km}\]

\item Hence, another equation is:
\[p = p_0 e^{\left(-\frac{z}{H} \right)}\]
\end{itemize}

\subsection{Equations}
\label{sec:orgec61488}
\[\frac{dp}{dz} = - \rho g = - \frac{p}{RT}g\]
\[\frac{dp}{p} = - \frac{g}{RT} \, dz\]
\[p = p_0 e^{\left(-\frac{z}{H} \right)}\]


\section{Energetics of atmospheric and oceanic motion}
\label{sec:org390f830}
As hydrostatic balance is dominant at large scales, horizontal motion is large and vertical motion is small in the atmosphere and oceans.

\begin{enumerate}
\item Unequal heating of the atmosphere and oceans generates gradients in potential energy.
\item Potential energy can be converted into kinetic energy by the movement of air and ocean water.
\item Kinetic energy is then dissipated by friction.
\end{enumerate}


\section{General circulation of the atmosphere}
\label{sec:orgf8b17ef}
Differential solar radiative heating causes a temperature gradient across the surface from the equator to the poles.
\\[0pt]

The north-south (meridional) circulation of the troposphere transport part of the heat from the equatorial zone to the polar regions. The tropical circulation is \textbf{dominated by the Hadley circulation}. There is a Hadley cell on either side of the intertropical convergence zone, located close to the equator.
\\[0pt]

Rising air in the intertropical convergence zone is replaced by \textbf{inflowing air (convergence)} at the surface. \textbf{Outflowing air (divergence)} in the upper troposphere sinks about \(30^{\circ} N\) and \(30^{\circ} S\), completing the circulation.

\subsection{Hadley cell}
\label{sec:org5c16e32}
Hot air rises in the intertropical convergence zone (ITCZ) due to shortwave absorption and sensible and latent heating. The hot air spreads polewards and cools by longwave emission.

\subsection{Ferrel cell}
\label{sec:org97fe257}
Warm subtropical air moves polewards and rises over the cold polar air at the polar front and recirculates towards the equator in the upper atmosphere.

\subsection{Polar cell}
\label{sec:orgd0efb0d}
Weak surface cold-air outbreaks.


\section{Distribution of surface winds}
\label{sec:org0ac6138}

\subsection{No Coriolis effect}
\label{sec:orgd4923ef}
\begin{center}
\includegraphics[width=.9\linewidth]{./images/surface-wind-distribution.png}
\end{center}

Surface winds blow out of the \textbf{high-pressure zones at the poles and at \(30^{\circ} N\) and \(30^{\circ} S\),} and blows towards the \textbf{low-pressure zones at the equator and in the mid-latitudes}.

\subsection{Coriolis effect}
\label{sec:org9c9a415}
\begin{center}
\includegraphics[width=.9\linewidth]{./images/surface-wind-distribution-coriolis-effect.png}
\end{center}

The winds are deflected to the \textbf{right} in the \textbf{Northern Hemisphere} and are deflected to the \textbf{left} in the \textbf{Southern Hemisphere}.


\section{Unstable polar front}
\label{sec:org6c92549}
The polar front is unstable and only lasts for a few days. In the process, anticyclones (\emph{high-pressure} systems at the surface) are usually formed upstream, and extratropical cyclones (\emph{low-pressure} systems at the surface) in the downstream.


\section{Wind-driven currents}
\label{sec:org2dcd03a}
\begin{itemize}
\item Apart from the two tides that happen daily as a result of the \textbf{gravitational pull} of the Moon and the Sun, ocean currents are driven almost entirely by \textbf{surface wind drag}.
\item Trade winds create warm easterly equatorial currents.
\item Warm currents flow polewards along western boundaries, and are indirectly created by westerly winds. (\textbf{Gulf Stream} and \textbf{Kuroshio}).
\item Deeper cold currents on the eastern boundaries close the circulation. (\textbf{California} and \textbf{Canary} currents).
\end{itemize}


\section{Possible conditions for condensation}
\label{sec:orgf8c2c0b}
\begin{enumerate}
\item Cooling at constant specific humidity and pressure. Vapour pressure \(e\) is constant while saturated vapour pressure \(e_s(T)\) falls as temperature \(T\) falls. Eventually, vapour pressure \(e >\) saturated vapour pressure \(e_s\).
\item Mixing of unsaturated air. This can sometimes result in supersaturation because saturated vapour pressure \(e_s\) decreases exponentially as temperature \(T\) falls, but internal energy (temperature \(T\)) and water vapour (vapour pressure \(e\)) mix linearly.
\item Adiabatic ascent of moist air. Adiabatic means that there is no net input of heat. As air ascends, temperature \(T\) falls due to adiabatic expansion. Vapour pressure \(e\) does not change much, but saturated vapour pressure \(e_s\) decreases exponentially. Eventually, vapour pressure \(e >\) saturated vapour pressure \(e_s\).
\end{enumerate}

\newpage

\section{Precipitation}
\label{sec:org1696cf4}
\begin{itemize}
\item Clouds are water droplets or ice crystals suspended in air currents.
\item They can evaporate or sublimate with time.
\item When the particles become too massive, they fall under gravity leading to "precipitation".
\end{itemize}

\subsection{Types of precipitation}
\label{sec:org39cca12}

\subsubsection{Rain}
\label{sec:org007a1aa}
Falling water droplets.

\subsubsection{Snow}
\label{sec:org47fbcbb}
Falling ice crystals (order-six symmetry; fluffy clumps).

\subsubsection{Graupel}
\label{sec:orgfb3bfd4}
Ice frozen onto falling snow forming soft pellets.

\subsubsection{Sleet}
\label{sec:org6d5c2b8}
Mixture of rain and snow.

\subsection{Characterisation of rainfall or snowfall}
\label{sec:orgb99394a}
\begin{itemize}
\item Intensity (\(\unit{mm.h^{-1}})\))
\item Area coverage (\(\unit{km^{-2}}\))
\item Duration (\(\unit{h}\))
\item Frequency (days per month)
\end{itemize}

\newpage

\section{Rainy regions}
\label{sec:orge7e92ba}
Rainy regions have \textbf{converging air}.

\subsection{Examples}
\label{sec:org8bb1f36}
\begin{itemize}
\item Intertropical convergence zone.
\item Mid-latitude storm tracks where activity is high.
\item \textbf{Windward slopes} of mountains, like the Himalayan foothills during the southwest monsoon.
\end{itemize}


\section{Deserts}
\label{sec:orgd627d08}
Deserts have \textbf{diverging air}.

\subsection{Examples}
\label{sec:orgf19d550}
\begin{itemize}
\item Subtropical zones, especially on the western side of continents where there are cold currents offshore.
\item \textbf{High-latitude or high-altitude regions}, as cold air has less moisture. Some examples include Siberia, Tibet and Antarctica.
\end{itemize}


\section{Collision of India with Asia}
\label{sec:orge7cb7a3}
\begin{itemize}
\item The Tibetan plateau has an area of 2 million \(\unit{km^{2}}\), with an average elevation of above \(\qty{5}{km}\).
\item Subduction processes do no cause "sudden" changes in global high terrain but continent-continent collisions do (such as the collision between India and Asia over the last 50 million years).
\end{itemize}

\newpage

\section{Ocean gateway hypothesis}
\label{sec:orgbe7761d}
\begin{itemize}
\item The opening of Drake's Passage between South America and Antarctica about 35 million years ago may have decreased oceanic poleward heat transport.
\item After opening, a strong Antarctic circumpolar current started which might have enhanced the glaciation in the Antarctic continent.
\item The closure of Isthmus of Panama about 3 million years ago caused increased poleward heat transport.
\item It might have provided more moisture to the North Atlantic region.
\item This increase in moisture may have helped start the major Northern Hemisphere glaciations around 2.5 million years ago.
\end{itemize}


\section{Eons}
\label{sec:org82eb52e}

\subsection{Precambrian}
\label{sec:orgc56b274}
4.54 billion years ago - 541 million years ago.

\subsubsection{Hadean era}
\label{sec:org4156a28}
4.54 billion years ago - 4.0 billion years ago.

\subsubsection{Archean era}
\label{sec:orgfcd6fee}
2.5 billion years ago - 4.0 billion years ago.

\subsubsection{Proterozoic era}
\label{sec:org5f3a3e0}
2.5 billion years ago - 541 million years ago.

\subsection{Phanerozoic}
\label{sec:org5a025f6}
541 million years ago - present time.


\section{Global warming}
\label{sec:org954a4e7}

\subsection{What causes stratospheric cooling to occur more rapidly than surface warming?}
\label{sec:org9ceb488}
\begin{itemize}
\item The stratosphere is thinner and has a low density of air molecules.
\item The troposphere contains 75\% of the mass of the atmosphere but only a small fraction of its volume.
\item More greenhouse gases in the troposphere traps more heat.
\item More greenhouse gases in the stratosphere results in molecules emitting more heat. However, the low density of air molecules result in the heat not transferring to other air molecules in the stratosphere. Instead, the heat emitted is lost to space.
\item As a result, the presence of more greenhouse gases means that more heat is lost to space.
\end{itemize}

\subsection{Why does land, despite its substantial mass, store relatively little of the excess heat?}
\label{sec:org85ba2ce}
\begin{itemize}
\item Sunlight penetrates and heats the upper tens of meters of the ocean, especially in the tropics, where the Sun's radiation arrives from a high angle. Winds blowing across the ocean's surface stir the upper layers and mix solar heat as deep as 100 meters.
\item In contrast, even though tropical and subtropical landmasses generally become very hot under the strong sunlight, they are not capable of storing much heat because heat is conducted down into soil or rock at very slow rates.
\end{itemize}

\newpage

\subsection{Heat stored in the ocean causes its water expand}
\label{sec:org66d4f2d}
\begin{itemize}
\item Thermal expansion of the oceans causes thermosteric sea level rise.
\item If an object is heated, its atoms vibrate faster and spread out, causing the object to expand. When it cools, the atoms slow down and the object shrinks.
\item Steric sea-level changes = Thermosteric changes due to changes in ocean's temperature + Halosteric changes due to variations in salt content (or salinity) of seawater.
\item Global mean sea-level rise was \(\qty{3.0}{mm.yr^{-1}}\) from 1993 to 2016.
\end{itemize}

\subsection{Possible causes of global warming}
\label{sec:orge4cefea}

\subsubsection{Solar forcing: changes in the solar energy (not the cause)}
\label{sec:org2d9e0fa}
\begin{itemize}
\item The Sun can influence Earth's climate, but it isn't responsible for global warming.
\item There has been no upward trend in the amount of solar energy reaching Earth.
\item If the Sun were responsible for global warming, we would expect to see warming throughout all layers of the atmosphere. But what we see is warming at the surface and cooling in the stratosphere.
\end{itemize}

\subsubsection{Orbital forcing: changes in the Earth's orbit (not the cause)}
\label{sec:org71dd168}
\begin{itemize}
\item Milankovitch (orbital) cycles are the collective effect of three periodic variations in the Earth's orbit (obliquity, eccentricity and axial precession) on the Earth's climate. They are the driving force behind glacial-interglacial cycles.
\item Milankovitch cycles operate on long time scales, ranging from tens of thousands to hundreds of thousands of years, and they have not significantly changed the amount of solar energy absorbed by Earth over the last 150 years.
\item If there were no human influences on climate, Earth's current orbital positions within the Milankovitch cycles predict our planet should be cooling, not warming, continuing a long-term cooling trend that began 6,000 years ago.
\end{itemize}

\subsubsection{Increasing CO\textsubscript{2} levels}
\label{sec:org5647660}
\begin{itemize}
\item Direct measurements by NOAA at Mauna Lao Observatory in Hawaii from 1958 to present.
\item The annual rise and fall of CO\textsubscript{2} levels is caused by seasonal cycles in photosynthesis on a massive scale.
\item Despite these small seasonal fluctuations, the overall trend shows that CO\textsubscript{2} is increasing at a roughly linear rate in the atmosphere.
\item The atmosphere's CO\textsubscript{2} content has increased by 50\% in less than 200 years.
\item The Earth once had higher CO\textsubscript{2} than now.
\end{itemize}

\subsubsection{Volcanic activities: part of tectonic forcing (not the cause)}
\label{sec:orged5e47f}
\begin{itemize}
\item Volcanoes emit CO\textsubscript{2} through eruptions and degassing.
\item Volcanoes are a major source for restoring CO\textsubscript{2} lost from the atmosphere and oceans by silicate weathering, carbonate deposition, and organic carbon burial.
\item Volcanoes can have a cooling effect.
\item Volcanic eruptions often produce volcanic ash and aerosol particles. Volcanic aerosols reflect sunlight back into space, blocking solar radiation.
\item The catastrophic eruption of Mount Pinatubo in 1991 ejected enormous amount of aerosol particles, which reflected so much incoming sunlight that global surface temperatures cooled off for two years.
\end{itemize}

\newpage

\subsubsection{Anthropogenic activities (the cause)}
\label{sec:orgcbbd25d}

\begin{center}
\begin{tabular}{|m{13em}|m{10em}|}
\hline
CO\textsubscript{2} emitters & Billion tons per year (\(\unit{Gt.yr^{-1}}\))\\[0pt]
\hline
Global volcanic emissions (highest preferred estimate) & 0.3 (Gerlach Eos, 2011) 0.6 (Burton et al., 2013)\\[0pt]
\hline
Human activities (fossil fuels and land use change) & Roughly 40\\[0pt]
\hline
\end{tabular}
\end{center}

\begin{itemize}
\item Annually, human activities produce roughly 100 times the carbon dioxide of Earth's volcanic eruptions.
\item Human influence has warmed the climate at an unprecedented rate in at least the last 2000 years.
\end{itemize}


\section{Uplift of high mountains and their effect}
\label{sec:orgb6bcb11}
\begin{itemize}
\item Mountains and plateaus have steep slopes.
\item Mass wasting processes, including rock slides, falls, and water-saturated debris flows, can move rock and expose fresh bedrock.
\item Mountain glaciers grind against underlying rock, creating smaller rock grains with larger surface areas for further chemical interactions.
\item Steep slopes also serve as precipitation hotspots on the upwind side, promoting chemical weathering.
\item Hence, the uplift of high mountains such as the Himalayas enhanced chemical weathering and removed more atmospheric CO\textsubscript{2}.
\end{itemize}

\subsection{Uplift of high mountains is not the cause of global cooling}
\label{sec:org7e3be25}
\begin{itemize}
\item The rate of CO\textsubscript{2} consumption decreased by 50\% between roughly 16 and 5.3 million years ago, especially in the Indus system.
\item Falling chemical weathering fluxes during a period of global cooling refutes the idea that the Himalayan-Tibetan Plateau uplift drove the Neogene global cooling.
\end{itemize}


\section{Excess CO\textsubscript{2}}
\label{sec:orgcf8815a}
\begin{itemize}
\item CO\textsubscript{2} remains in the atmosphere long after emissions.
\item Without the sequestration of carbon by the land and ocean, the level of CO\textsubscript{2} in the atmosphere would now be around 600 parts per million, with an associated warming about double the current level.
\end{itemize}


\section{CO\textsubscript{2} and the oceans}
\label{sec:orgdf645d7}
\begin{itemize}
\item CO\textsubscript{2} can be dissolved in seawater, like how it dissolves in a can of soda.
\item CO\textsubscript{2} can also be released from seawater, much like the way CO\textsubscript{2} is released when soda fizzes.
\item CO\textsubscript{2} emissions acidify the Earth's oceans.
\end{itemize}

\subsection{CO\textsubscript{2} flux across the air-sea interface in 2000}
\label{sec:org87cd854}
\begin{itemize}
\item Colder oceans absorb CO\textsubscript{2} and warmer oceans release CO\textsubscript{2} into the atmosphere because CO\textsubscript{2} is more soluble in colder water
\end{itemize}
\begin{center}
\includegraphics[width=.9\linewidth]{./images/co-2-flux-map.png}
\end{center}

\subsection{Marine carbon cycle}
\label{sec:org049b3e4}
\begin{center}
\includegraphics[width=.9\linewidth]{./images/marine-carbon-cycle.png}
\end{center}

\subsection{Why can't oceans be an immediate CO\textsubscript{2} sink?}
\label{sec:org444b80f}
\begin{itemize}
\item If the atmospheric concentration changes by 10\%, the concentration in solution in the water changes by only one-tenth of this, which is 1\%.
\item This change will occur quite rapidly in the upper waters of the ocean, which is the top \(\qty{100}{m}\) or so.
\item Absorption in the lower levels in the ocean takes longer.
\item The mixing of surface water with water at lower levels takes up several hundred years, or for the deep ocean, over a thousand years.
\end{itemize}


\section{Hazards around Singapore}
\label{sec:org74c90dc}
\begin{itemize}
\item Seismic and tsunami hazard
\item Volcanic hazard
\item Typhoon and cyclone hazard
\item Sea level rise
\end{itemize}


\section{Sea-level driving processes}
\label{sec:org0b9b3c7}
Sea level varies in time and in space.

\subsection{Global driving processes}
\label{sec:org8624649}

\subsubsection{Volume problem}
\label{sec:org6842f48}
An increase in ocean temperatures result in an increase in the volume of ocean water, which contributes 40\% of the global sea level rise.

\subsubsection{Mass problem}
\label{sec:org91fc337}
\begin{itemize}
\item An increase in global temperatures result in the melting of ice sheets, which increase the mass of water in the oceans and hence the volume of water in the ocean, which contributes 30\% of the global sea level rise.
\item Greenland will completely be free of ice if global temperatures reach \(\qty{2}{\degreeCelsius}\) above pre-industrial levels.
\end{itemize}

\subsection{Regional driving processes}
\label{sec:orgfa13854}

\subsubsection{Pro-glacial forebulge}
\label{sec:org6dd47e8}
\begin{itemize}
\item A pro-glacial forebulge is formed due to the weight of an ice sheet pressing down on the earth, lifting the surrounding land around the ice sheet up.
\item The melting of the ice sheet will result in the forebulge sinking due to the reduction of mass pressing down on the earth, which results in the earth under the ice sheet rising, and the forebulge sinking.
\item This makes the sea level rise much quicker in the area as the land is sinking as well when the sea level is rising.
\end{itemize}

\newpage

\subsubsection{Gravitational attraction of ice sheets}
\label{sec:orgeb42e24}
\begin{itemize}
\item The density of ice sheets are much larger than that of the earth, which results in stronger gravitational attraction in areas with ice sheets.
\item Thus, the water is pulled towards the ice sheet.
\item When the ice sheet beside the ocean melts, there is a fall in sea levels in the area due to the reduction in the gravitational attraction of the area.
\end{itemize}

\subsection{Local driving processes}
\label{sec:org0652df2}

\subsubsection{Subsidence}
\label{sec:org151f3b4}
\begin{itemize}
\item The land near the coastline in some areas may have groundwater, which results in the land sinking.
\item This results in a greater sea level rise in the area as the land is sinking when the sea level is rising.
\end{itemize}

\subsubsection{Tectonic uplift}
\label{sec:orga389671}
\begin{itemize}
\item The oceanic plate in some coastlines subduct under the continental plate the land sits on, which results in the continental plate rising and hence the land rises.
\item This results in a lower sea level rise in the area as the land is rising while the sea level is rising.
\end{itemize}

\subsubsection{Ocean dynamics}
\label{sec:org234e577}
The ocean winds can shift create larger waves in certain areas, which result in a greater sea level rise in some areas as compared to others.

\newpage

\section{Sea levels}
\label{sec:org3c4de3d}

\subsection{Last Glacial Maximum - Present day}
\label{sec:org5078135}
\begin{itemize}
\item At the Last Glacial Maximum, Northern Europe and America were covered by vast ice sheets is roughly \(\qty{3}{km}\) thick.
\item As a result of the water being locked up, global sea levels were roughly \(\qty{130}{m}\) below present.
\item Adjustment of the land to the unloading of ice and redistribution of meltwater caused spatially variable sea-level changes that continue to this day.
\item Due to the melting of the large ice sheets since the Last Glacial Maximum, the Glacial Isostatic Adjustment process causes highly spatially variable sea-level histories across the globe.
\end{itemize}

\subsection{Holocene}
\label{sec:org81ba675}
Holocene sea levels in Singapore rose from roughly \(\qty{-20.7}{m}\) 9.5 thousand years ago to \(\qty{4}{m}\) high 5.2 thousand years ago before falling thereafter.

\subsection{Common Era sea levels}
\label{sec:org85f96f3}
It is very likely that the rates of sea-level rise emerged above pre-industrial rates by 1863, which is similar in timing to evidence for early ocean warming and glacier melt.

\subsection{Present sea levels}
\label{sec:org5cd5e7d}
\begin{itemize}
\item Global mean sea level increased by \(\qty{0.20}{m}\) between 1901 and 2018.
\item The rate of sea level rise is \textbf{accelerating}.
\item The rate was \(\qty{1.3}{mm.yr^{-1}}\) between 1901 - 1971, \(\qty{1.9}{mm.yr^{-1}}\), between 1971 - 2006, and was \(\qty{1.9}{mm.yr^{-1}}\) between 2006 - 2018.
\end{itemize}

\newpage

\subsection{Future sea levels}
\label{sec:orgae81873}
\begin{itemize}
\item Sea-level projections are provided for 5 Shared Socioeconomic Pathway (SSP) scenarios that consider driving processes for which we have at least medium confidence.
\item It is presented as likely ranges representing 17\textsuperscript{th} - 83\textsuperscript{rd} percentile representing a probability of at least 66\%.
\end{itemize}

\subsubsection{SSP1-1.9}
\label{sec:org831d2f0}
\begin{itemize}
\item Globally averaged surface air temperature over the period 1081 - 2100 is very likely (at least 90\% probability) to be higher by \(\qty{1.0}{\degreeCelsius} - \qty{1.8}{\degreeCelsius}\) compared to 1850 - 1900.
\item Net-zero CO\textsubscript{2} emissions around the middle of the century.
\end{itemize}


\subsubsection{SSP5-8.5}
\label{sec:org19bfdcc}
\begin{itemize}
\item Globally averaged surface air temperature over the period 2081 - 2100 is very likely to be higher by \(\qty{3.3}{\degreeCelsius} - \qty{5.7}{\degreeCelsius}\) compared to 1850 - 1900.
\item High reference scenario with no additional climate policy.
\item Emission levels as high as SSP 5 - SSP 8.5 are not obtained by Integrated Assessment Models (IAMs) under any of the SSPs other than the fossil fuelled SSP5 economic development pathway.
\end{itemize}

\subsubsection{SSP5-8.5 (low confidence)}
\label{sec:org3d8d1e4}
\begin{itemize}
\item Globally averaged surface air temperature over the period 2081 - 2100 is very likely to be higher by \(\qty{3.3}{\degreeCelsius} - \qty{5.7}{\degreeCelsius}\) compared to 1850 - 1900.
\item High reference scenario with no additional climate policy.
\item Emission levels as high as SSP 5 - SSP 8.5 are not obtained by Integrated Assessment Models (IAMs) under any of the SSPs other than the fossil fuelled SSP5 economic development pathway.
\item Integrates information from the Structured Expert Judgement and results from a simulation study that incorporates Greenland and Antarctica ice sheet processes for which we have low confidence.
\end{itemize}

\subsubsection{High impact processes}
\label{sec:orgf6023ea}
Higher amounts of sea-level rise before 2100 could be caused by earlier-than-projected disintegration of Antarctica through the abrupt, widespread onset of Marine Ice Sheet Instability and Marine Ice Cliff Instability.

\subsubsection{Projected exposure of coastal ecosystems to sea-level rise}
\label{sec:org628d508}
\begin{itemize}
\item Nearly all the world's mangrove forests and coral reefs and 40\% of tidal marshes will be subjected to rising sea level rates of \(\qty{7}{mm.yr^{-1}}\) by 2100 under a global mean warming scenario of \(\qty{3}{\degreeCelsius}\) above the 1850 - 1900 baseline (SSP2 - SSP 4.5).
\item Once these tipping points are reached, there is little prospect of reversal during the following century, and ecological and socio-economic consequences will be profound.
\end{itemize}

\subsection{Driving human migration}
\label{sec:orgcb46a48}
\begin{itemize}
\item In Southeast Asia, rising sea levels flooded the Sunda Shelf and reduced land area by over 50\%, resulting in segregation of local human populations.
\item Integrated paleogeographic and population genomic analysis demonstrates the earliest instance of forced human migration driven by sea-level rise.
\item Spikes in human population displacement coincides with occurrences of rapid sea-level rise.
\end{itemize}

\subsection{Sea-level tendency}
\label{sec:orgd6d6f79}

\subsubsection{Transgressive contacts (Positive tendency)}
\label{sec:org0251f7d}
A change from a marsh to a mudflat deposit (marsh retreat).

\subsubsection{Regressive contacts (Negative tendency)}
\label{sec:orgf67212c}
Replacement of mudflat by a marsh deposit (marsh expansion).

\subsection{Summary}
\label{sec:org05440c3}
\begin{enumerate}
\item Past and present records of sea level have varied in response to a wide range of boundary conditions and climate forcing and can serve as a valuable guide to projecting future sea-level rise and its uncertainty.
\item Ice mass loss from glaciers, Greenland and Antarctica is accelerating and will continue to lose mass throughout the 21st century under all considered SSP scenarios.
\item Sea level will continue to rise through 2100.
\item Higher amounts of sea-level rise before 2100 could be caused by Marine Ice Sheet Instability and Marine Ice Cliff Instability. Such processes could contribute more than one additional meter of sea level rise by 2100.
\end{enumerate}

\newpage

\section{Extremes}
\label{sec:org753ea0a}

\subsection{How does climate change affect temperature extremes?}
\label{sec:orge10e0fb}
\begin{enumerate}
\item A small shift of the mean by itself greatly enhances the probability of one extreme and greatly reduces the probability of the other extreme.
\item A small increase in the variance by itself results in huge enhancements in the probabilities of both extremes.
\item The symmetry of the probability distribution can also be altered such that only one extreme is adversely affected.
\end{enumerate}

\subsection{Why does climate change cause more frequent and intense rainfall?}
\label{sec:org637c664}
\begin{itemize}
\item At the global scale, column-integrated water vapour content increases roughly following the \textbf{Clausius-Clapeyron (C-C) relation}, with an increase of approximately \(\boldsymbol{7\%} \textbf{ per } \boldsymbol{\qty{1}{\degreeCelsius}}\) of global warming.
\item The intensification of heavy precipitation will follow the rate of increase in the maximum amount of moisture that the atmosphere can hold as it warms (high confidence), of about \(\boldsymbol{7\%} \textbf{ per } \boldsymbol{\qty{1}{\degreeCelsius}}\) of global warming.
\item Global mean precipitation and evaporation increase with global warming, but the estimated rate is model-dependent (very likely range of \(\boldsymbol{1 - 3\%} \textbf{ per } \boldsymbol{\qty{1}{\degreeCelsius}}\)).
\item The increase in water vapour leads to robust increases in precipitation extremes everywhere, with a magnitude that varies between 4\% and 8\% per \(\qty{1}{\degreeCelsius}\) of surface warming (thermodynamic contribution).
\end{itemize}

\newpage

\subsection{Climate change and extreme events}
\label{sec:orgc85f6d3}
\begin{itemize}
\item Future changes in \textbf{temperature} averages and extremes will be \textbf{similar}, while the future changes in \textbf{precipitation} averages and extremes can be \textbf{very different}.
\item Scientists cannot answer directly whether a particular event was caused by climate change, as extremes do occur naturally, and any specific weather and climate event is the result of a complex mix of human and natural factors. Instead, scientists quantify the relative importance of human and natural influences on the magnitude and probability of specific extreme weather events.
\item \textbf{Attribution} is done by estimating and comparing the probability or magnitude of the same type of event between the current climate, like the increases in greenhouse gas concentrations and other human influences, and an alternate world where the atmospheric greenhouse gases remained at pre-industrial levels.
\item Strong increases in probability and magnitude, attributable to human influence, have been found for many heatwaves and hot extremes around the world.
\item Attributable increases have also been found for some extreme precipitation events, including hurricane rainfall events, but these results can vary among events. In some cases, large natural variations in the climate system prevent attributing changes in the probability or magnitude of a specific extreme to human influence.
\item As the climate continues to warm, larger changes in probability and magnitude are expected and, as a result, it will be possible to attribute future temperature and precipitation extremes in many locations to human influences. Attributable changes may emerge for other types of extremes as the warming signal increases.
\item Human-caused climate change increases wildfire by intensifying its principal driving factor, heat. The heat of climate change dries out vegetation and accelerates burning. Non-climate factors also cause wildfires.
\end{itemize}

\newpage

\begin{itemize}
\item Evidence shows that human-caused climate change has driven increases in the area burned by wildfire in the forests of western North America. Across this region, the higher temperatures of human-caused climate change doubled burned area from 1984 to 2015, compared with what would have burned without climate change.
\item In other regions, wildfires are also burning wider areas and occurring more often. This is consistent with climate change, but analyses have not yet shown if climate change is more important than other factors. In Indonesia, intentional burning of rainforests for oil palm plantations and El Niño seem to be more important than long-term climate change.
\item 90 percent of the world's population, both rich and poor countries alike, will be exposed to one or more threats arising from global warming.
\item Far more people (about 600 million people) are seeing heavier precipitation than lighter precipitation (80 million).
\end{itemize}

\newpage

\section{Climate models}
\label{sec:org5d4cd46}

\subsection{Nobel Laureates in climate science (2021)}
\label{sec:orgb01bbdb}
\begin{enumerate}
\item Syukuro Manabe
\item Klaus Hasselmann
\end{enumerate}

\subsection{How it works}
\label{sec:org3e5fe69}
\begin{itemize}
\item The Earth is divided into grid boxes.
\item The governing equations are then calculated for every grid box.
\item The influence of vegetation and terrain is included in the calculation.
\item The atmosphere is divided into cubes, each with its own local climate.
\item The air in grid boxes interacts horizontally and vertically with other boxes.
\item Oceanic grid boxes model currents, temperature, and salinity.
\item Water in oceanic grid boxes interacts horizontally and vertically with other boxes.
\end{itemize}

\subsection{Ocean models vs atmosphere models}
\label{sec:org5e68148}
\begin{center}
\begin{tabular}{lll}
\textbf{Property} & \textbf{Ocean models} & \textbf{Atmosphere models}\\[0pt]
\hline
Heating & From top & From bottom\\[0pt]
Memory & Long & Short (except stratosphere)\\[0pt]
Density variations & Small & Large\\[0pt]
Density profile & Increases with depth & Decreases with altitude\\[0pt]
Obstruction & Yes, strong boundary currents & No\\[0pt]
\end{tabular}
\end{center}

\newpage

\subsection{Challenges}
\label{sec:orgcf6a7d7}
\begin{itemize}
\item Unknown future greenhouse gas concentration
\item Natural variability of the climate, and hence there is a lack of long data records
\item Limitation in understanding the full climate system
\item High-resolution global climate modelling is computationally expensive
\item Stubborn mean-state biases
\end{itemize}

\subsection{Model bias}
\label{sec:org955e50c}
\[\text{Model bias} = \text{model simulation} - \text{observations}\]

\begin{itemize}
\item Substantial deviations from observed climate on regional and local scales are possible
\item Climate models are a simplification of the climate system, and the large-scale grid cells may not represent processes that happen on a smaller scale than the area of grid cells.
\end{itemize}

\subsubsection{Sources of bias}
\label{sec:orge528598}
\begin{itemize}
\item Low horizontal resolution of the model
\item Inappropriate model physics
\item Error in the input data or the lack of observations
\item Model structure
\item Error in the initial condition for weather forecasting
\item Unknown future conditions (for climate change projections)
\end{itemize}
\end{document}
