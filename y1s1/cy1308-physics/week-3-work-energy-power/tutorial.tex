% Created 2023-09-08 Fri 12:49
% Intended LaTeX compiler: pdflatex
\documentclass[11pt]{article}
\usepackage[utf8]{inputenc}
\usepackage[T1]{fontenc}
\usepackage{graphicx}
\usepackage{longtable}
\usepackage{wrapfig}
\usepackage{rotating}
\usepackage[normalem]{ulem}
\usepackage{amsmath}
\usepackage{amssymb}
\usepackage{capt-of}
\usepackage{hyperref}
\usepackage{siunitx}
\author{Hankertrix}
\date{\today}
\title{Work Energy Power Tutorial}
\hypersetup{
 pdfauthor={Hankertrix},
 pdftitle={Work Energy Power Tutorial},
 pdfkeywords={},
 pdfsubject={},
 pdfcreator={Emacs 29.1 (Org mode 9.6.6)}, 
 pdflang={English}}
\begin{document}

\maketitle
\setcounter{tocdepth}{2}
\tableofcontents


\section{Question 1}
\label{sec:org8bf0041}

\subsection{(a)}
\label{sec:orgd1a7f95}
Let the net force on the crate be \(F_net\), the frictional force on the crate be \(f\) and the normal contact force on the crate be \(F_N\):
\begin{align*}
F_{net} &= mg \sin \theta - f \\
&= mg \sin \theta - F_n \mu_k \\
&= mg \sin \theta - mg \cos \theta \mu_k \\
&= mg (\sin \theta - \mu_k \cos \theta)
\end{align*}

By Newton's Second Law:
\[F_{net} = ma\]
\[ma = mg (\sin \theta -\mu_k \cos \theta)\]
\[a = g (\sin \theta - \mu_k \cos \theta)\]
\[a = 9.81 (\sin 25 \unit{\degree} - 0.19 \cos 25 \unit{\degree})\]
\[a = 2.456618063\]
\[a = \SI{2.46}{\unit{m.s^{-2}}} \text{ (3 s.f)}\]

\subsection{(b)}
\label{sec:org5f5238a}
Getting the time taken for the crate to travel the distance:
\[s = ut + \frac{1}{2}at^2\]
\[8.15 = 0 + \frac{1}{2}(2.456618063)t^2\]
\[8.15 \times 2 = 2.456618063t^2\]
\[t^2 = 6.635138056\]
\[t = 2.575876173\]

Getting the final speed of the crate:
\[v = v_0 + at\]
\[v = 0 + 2.456618063 \times 2.575876173\]
\[v = 6.327943934\]
\[v = \SI{6.33}{\unit{m.s^{-2}}} \text{ (3 s.f)}\]


\section{Question 2}
\label{sec:org779d384}

\subsection{(a)}
\label{sec:orgdc86771}
First, the force \(F\) needs to overcome the frictional force due to the ground (\(f_g\)):
\begin{align*}
f_g &= mg \mu_s \\
&= (5 + 3)(9.8)(0.6) \\
&= \qty{47.04}{\unit{N}}
\end{align*}

Next, the force \(F\) needs to overcome the frictional force on the \(\qty{3.0}{\unit{kg}}\) \(f_{3kg}\):
\begin{align*}
f_{3kg} &= mg \mu_s \\
&= 3(9.8)(0.6) \\
&= \qty{17.64}{\unit{N}}
\end{align*}

The force \(F\) also needs to overcome the frictional force from the movement of the \(\qty{5.0}{\unit{kg}}\) causing the \(\qty{3.0}{\unit{kg}}\) to move backwards (\(f_{mov}\)):
\begin{align*}
f_{mov} = &= mg \mu_s \\
&= 3(9.8)(0.6) \\
&= \qty{17.64}{\unit{N}}
\end{align*}

Thus, the force \(F\) is:
\begin{align*}
F &= f_g + f_{3kg} + f_{mov} \\
&= 47.4 + 17.64 + 17.64 \\
&= \qty{82.32}{\unit{N}} \\
&= \qty{82.3}{\unit{N}} \text{ (3.s.f)}
\end{align*}

\newpage

\subsection{(b)}
\label{sec:org70c4dde}
Finding the force when it is 10\% greater:
\begin{align*}
F &= 1.1(82.32) \\
&= \qty{90.552}{\unit{N}}
\end{align*}

Since the blocks start to move, we will need to use kinetic friction, \(\mu_k\). The blocks are also joined together with a rope, so the acceleration of the two blocks will be the same.
\\[0pt]

From part \((a)\):
\[F_{net} = F - (f_g + f_{3kg} + f_{mov})\]
\[F_{net} = 90.552 - ((5 + 3)(9.8)(0.4) + (3)(9.8)(0.4) + (3)(9.8)(0.4))\]
\[F_{net} = 90.552 - ((5 + 3)(9.8)(0.4) + (3)(9.8)(0.4) + (3)(9.8)(0.4))\]
\[(5 + 3)a = 90.552 - ((5 + 3)(9.8)(0.4) + (3)(9.8)(0.4) + (3)(9.8)(0.4))\]
\[8a = 35.672\]
\[a = 4.459\]
\[a = \qty{4.46}{\unit{m.s^{-2}}} \text{ (3 s.f)}\]

\section{Question 3}
\label{sec:orge33dc93}

\subsection{(a)}
\label{sec:orgf561d8f}
\[F_{net} = F - f\]
\[F_{net} = 650 - 65(9.8)(0.18) - 125(9.8)(0.18)\]
\[ma = 314.84\]
\[190a = 314.84\]
\[a = 1.657052632\]
\[a = \qty{1.66}{\unit{m.s^{-2}}} \text{ (3 s.f)}\]

\subsection{(b)}
\label{sec:org0da37d7}
Let the contact force that acts on the first block be \(F_c\):
\[F - F_c - f = ma\]
\[650 - F_c - 65(9.8)(0.18) = 65(1.657052632)\]
\[F_c = 650 - 107.7084211 - 114.66\]
\[F_c = 427.6315789\]
\[F_c = \qty{428}{\unit{N}}\]

\subsection{(c)}
\label{sec:org4cd32b4}
\[F_{net} = F - f\]
\[F_{net} = 650 - 65(9.8)(0.18) - 125(9.8)(0.18)\]
\[ma = 314.84\]
\[190a = 314.84\]
\[a = 1.657052632\]
\[a = \qty{1.66}{\unit{m.s^{-2}}} \text{ (3 s.f)}\]

Let the contact force that acts on the first block be \(F_c\):
\[F - F_c - f = ma\]
\[650 - F_c - 125(9.8)(0.18) = 125(1.657052632)\]
\[F_c = 650 - 220.5 - 207.131579\]
\[F_c = 222.368421\]
\[F_c = \qty{222}{\unit{N}} \text{ (3 s.f)}\]


\section{Question 4}
\label{sec:orgf2308fb}

\subsection{(a)}
\label{sec:orgca0ba56}
Let \(l\) be the distance moved by the object with mass \(m_1\). When the object with mass \(m_1\) moves a distance \(l\), \(P_2\) moves a distance \(l\) towards \(P_1\). At the same time, the object with mass \(m_2\) also moves a distance \(l\) towards \(P_2\). As such, the object with mass \(m_2\) will effectively be moving \(2l\) when the object with mass \(m_1\) moves a distance \(l\), which would mean that \(2a_1 = a_2\).

\subsection{(b)}
\label{sec:orga576ac8}
The tension in the string for \(P_1\) would be:
\[F_{net} = m_1 g - T_1\]
\[m_1 g - T_1 = m_1 a_1\]
\[T_1 = m_1 g - m_1 a_1\]

The tension in the string for \(P_2\) would be:
\[F_{net} = T_2\]
\[F_{net} = m_2 a_2\]
\[T_2 = m_2 a_2\]

\subsection{(c)}
\label{sec:orgb0712d9}
The net force on the block with mass \(m_1\) is:
\[F_{net} = m_1 g - T_1\]

Since the tension in \(P_1\) is the force on \(P_2\), which is twice of the tension in the string of \(P_2\):
\[m_1 a_1 = m_1 g - 2T_2\]
\[m_1 a_1 = m_1 g - 2m_2 a_2\]

Since \(a_1 = 2a_2\):
\[m_1 a_1 = m_1 g - 4m_2 a_1\]
\[a_1 (m_1 + 4m_2) = m_1 g\]
\[a_1 (m_1 + 4m_2) = m_1 g\]
\[a_1 = \frac{m_1 g}{m_1 + 4m_2}\]

Since \(a_1 = 2a_2\):
\[a_2 = \frac{2m_1 g}{m_1 + 4m_2}\]


\section{Question 5}
\label{sec:orgcde62d8}

\subsection{(a)}
\label{sec:org259a7ec}
Using the conservation of momentum:
\[m_1 v_1 + m_2 v_2 = 0\]
\[m_1 v_1 = - m_2 v_2\]
\[0.500(4.00) = -3.0v\]
\[v_2 = -\frac{2}{3}\]
\[v_2 = \qty{- 0.667}{\unit{m.s^{-1}}}\]

\subsection{(b)}
\label{sec:orgf80f0d3}
By the conservation of energy, the potential energy of the block must have been fully converted into the kinetic energy of the block and the wedge:
\[E_p = \text{KE}_{block} + \text{KE}_{wedge}\]
\[m_1gh = \frac{1}{2}m_1 v_1^2 + \frac{1}{2}m_2 v_2^2\]
\[(0.500)(9.8)h = \frac{1}{2} (0.5) (4)^2 + \frac{1}{2} (3) \left( \frac{2}{3} \right)^2\]
\[4.9h = \frac{14}{3}\]
\[h = \frac{20}{21}\]
\[h = \qty{0.952}{\unit{m}}\]

\newpage

\section{Question 6}
\label{sec:orga143417}

By the conservation of energy, the potential energy gained by the spring must be equal to the potential energy of the climber, hence:
\[\frac{1}{2}kx^2 = mgh\]
\[\frac{1}{2}kx^2 = mg(2l + x)\]
\[kx^2 = 2mg(2l + x)\]
\[kx^2 = 4mgl + 2mgx\]
\[kx^2 - 2mgx - 4mgl = 0\]
\[x = \frac{2mg \pm \sqrt{(-2mg)^2 - 4(k)(-4mgl)}}{2k}\]
\[x = \frac{2mg \pm \sqrt{4m^2 g^2 + 16kmgl}}{2k}\]
\[x = \frac{2mg \pm \sqrt{4m^2 g^2 \left(1 + \frac{4kl}{mg} \right)}}{2k}\]
\[x = \frac{2mg \pm 2mg\sqrt{1 + \frac{4kl}{mg}}}{2k}\]
\[x = 2mg \left(\frac{1 \pm \sqrt{1 + \frac{4kl}{mg}}}{2k} \right)\]
\[x = mg \left(\frac{1 \pm \sqrt{1 + \frac{4kl}{mg}}}{k} \right)\]
\[x = \frac{mg}{k} \left(1 \pm \sqrt{1 + \frac{4kl}{mg}} \right)\]

Since \(x\) is always positive:
\[x = \frac{mg}{k} \left(1 + \sqrt{1 + \frac{4kl}{mg}} \right) \textbf{ (Shown)}\]


\section{Question 7}
\label{sec:orga8a6725}

\subsection{(a)}
\label{sec:orga796a7e}

\subsubsection{(i)}
\label{sec:orgf1698a3}
The velocity of the air relative to the cyclist would be:
\[v_r = v + w\]

The power developed by the cyclist would be:
\[P = \frac{WD}{t}\]
\[P = F \left(\frac{x}{t} \right)\]
\[P = Fv\]
\[P = (k(v + w)^2)v\]
\[P = kv(v + w)^2\]

\newpage

\subsubsection{(ii)}
\label{sec:org6c4cf57}
The velocity of the air relative to the cyclist would be:
\[v_r = \sqrt{v^2 + w^2}\]

Let the angle between the velocity of the air and the velocity of the cyclist be \(\theta\).
\[\cos \theta = \frac{A}{H}\]
\[\cos \theta = \frac{v}{\sqrt{v^2 + w^2}}\]

Resolving the velocity of the cyclist in the direction of the air resistance:
\begin{align*}
v \cos \theta &= v \frac{v}{\sqrt{v^2 + w^2}} \\
&= \frac{v^2}{\sqrt{v^2 + w^2}}
\end{align*}

The power developed by the cyclist would be:
\[P = \frac{WD}{t}\]
\[P = F \left(\frac{x}{t} \right)\]
\[P = Fv\]
\[P = (k(\sqrt{v^2 + w^2})^2) \left( \frac{v^2}{\sqrt{v^2 + w^2}} \right)\]
\[P = k(v^2 + w^2) \left( \frac{v^2}{\sqrt{v^2 + w^2}} \right)\]
\[P = kv^2\sqrt{v^2 + w^2}\]

\newpage

\subsection{(b)}
\label{sec:orgc5ae03d}

From \textbf{(aii)}, the power required to cycle at speed \(v\) in a cross wind of speed \(v\) is:
\begin{align*}
P_{cross} &= kv^2 \sqrt{v^2 + v^2} \\
&= kv^2 (\sqrt{2} v) \\
&= \sqrt{2}kv^3
\end{align*}

The power required to cycle at speed \(v\) in still air is:
\begin{align*}
P_{still} &= \frac{WD}{t} \\
&= F \left(\frac{x}{t} \right) \\
&= Fv \\
&= kv^2(v) \\
&= kv^3
\end{align*}

\[\frac{P_{cross}}{P_{still}} = \frac{\sqrt{2}kv^3}{kv^3}\]
\[\frac{P_{cross}}{P_{still}} = \sqrt{2}\]

Hence, the power required to cycle at speed \(v\) in a cross wind speed of speed \(v\) is \(\sqrt{2}\) greater than the power required to cycle at the same speed \(v\) in still air \textbf{(shown)}.
\end{document}