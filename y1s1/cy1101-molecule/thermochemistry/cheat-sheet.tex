% Created 2023-09-19 Tue 15:50
% Intended LaTeX compiler: pdflatex
\documentclass[11pt]{article}
\usepackage[utf8]{inputenc}
\usepackage[T1]{fontenc}
\usepackage{graphicx}
\usepackage{longtable}
\usepackage{wrapfig}
\usepackage{rotating}
\usepackage[normalem]{ulem}
\usepackage{amsmath}
\usepackage{amssymb}
\usepackage{capt-of}
\usepackage{hyperref}
\usepackage{siunitx}
\author{Hankertrix}
\date{\today}
\title{Thermochemistry Cheat Sheet}
\hypersetup{
 pdfauthor={Hankertrix},
 pdftitle={Thermochemistry Cheat Sheet},
 pdfkeywords={},
 pdfsubject={},
 pdfcreator={Emacs 29.1 (Org mode 9.6.6)}, 
 pdflang={English}}
\begin{document}

\maketitle
\setcounter{tocdepth}{2}
\tableofcontents

\newpage

\section{Definitions}
\label{sec:orgde4c3fd}

\subsection{Conservation of energy}
\label{sec:org398447e}
Energy cannot be created or destroyed; it can only be converted from one form to another.

\subsection{Thermal energy}
\label{sec:org41a01c8}
Thermal energy refers to the kinetic energy of molecular motion. It is measured by finding the \textbf{temperature} of an object.

\subsection{Heat}
\label{sec:orgb8698d8}
Heat refers to the amount of thermal energy transferred from one object to another as the result of a \textbf{temperature difference} between the two.

\subsection{First law of thermodynamics}
\label{sec:org92537c7}
The total internal energy \(E\) of an isolated system is constant.

\[\Delta E  = E_{final} - E_{initial}\]

\subsection{State function}
\label{sec:org9bae28f}
A state function is a function or property whose value depends only on the present state or condition of the system, not on the path used to arrive at that state.

\subsection{Expansion work}
\label{sec:org830884f}
Expansion work is the work done as the result of a volume change in the system.

\subsection{Enthalpy}
\label{sec:orgbe1a231}
Enthalpy refers to the total heat content of a system. It is equal to the internal energy of the system plus the work done on the system. It is also a state function.

\[\Delta H = \Delta E + w\]
\[\Delta H = \Delta E + P \Delta V\]

\newpage

\subsection{Thermodynamic standard state}
\label{sec:orga78b518}
The thermodynamic standard state is the most stable form of a substance. The conditions are:
\begin{enumerate}
\item Pressure of \(\qty{1}{\unit{atm}}\)
\item Temperature of \(\qty{25}{\unit{\degreeCelsius}}\)
\item Concentration of \(\qty{1}{\unit{M}}\) for all substances in solution
\end{enumerate}

Values at this state are denoted with a superscript circle, like \(\Delta H^\circ, \Delta S^\circ\) and \(\Delta G^\circ\).

\subsection{Enthalpy change of fusion (\(\Delta H_{fusion}\))}
\label{sec:org057dae0}
The enthalpy change of fusion is the amount of heat required to \textbf{melt} a substance \textbf{without changing its temperature}.

\subsection{Enthalpy change of vaporisation (\(\Delta H_{vap}\))}
\label{sec:org9b09ea7}
The enthalpy change of vaporisation is the amount of heat required to \textbf{vaporise} a substance \textbf{without changing its temperature}.

\subsection{Enthalpy change of sublimation (\(\Delta H_{subl}\))}
\label{sec:orgd4652ac}
The enthalpy change of sublimation is the amount of heat required to convert a substance from a \textbf{solid to a gas without going through a liquid phase}.
\\[0pt]

At constant temperature:
\[\Delta H_{subl} = \Delta H_{fusion} + \Delta H_{vap}\]

\subsection{Exothermic reaction}
\label{sec:orgaddae7f}
An exothermic reaction is a reaction that releases heat, which means its enthalpy change is negative, i.e. \(\Delta H < 0\).

\subsection{Endothermic reaction}
\label{sec:org0a4625e}
An endothermic reaction is a reaction that absorbs heat, which means its enthalpy change is positive, i.e. \(\Delta H > 0\).

\subsection{Heat capacity (\(C\))}
\label{sec:org7def809}
Heat capacity refers to the amount of heat necessary to raise the temperature of an object or substance a given amount.

\[C = \frac{q}{\Delta T}\]
\[q = C \times \Delta T\]

\subsection{Specific heat capacity (\(c\))}
\label{sec:org8c92e57}
Specific heat capacity refers to the amount of heat required to raise the temperature of \(\qty{1}{\unit{g}}\) of a substance by \(\qty{1}{\unit{\degreeCelsius}}\).

\[q = mc \Delta T, \ \ \text{where } m \text{ is the mass of the substance in } \unit{g}\]

\subsection{Molar heat capacity (\(C_m\))}
\label{sec:orgbb7f6eb}
Molar heat capacity is the amount of heat necessary to raise the temperature of \(\qty{1}{\unit{mol}}\) of a substance by \(\qty{1}{\unit{\degreeCelsius}}\).

\[q = C_m \times n \times \Delta T, \ \ \text{where } n \text{ is the number of moles of the substance}\]

\subsection{Hess's law}
\label{sec:org24eb0df}
Hess's law states that the overall enthalpy change for a reaction is equal to the sum of the enthalpy changes for the individual steps in the reaction.
\\[0pt]

Essentially:
\[\Delta H_{overall} = \Delta H_{1} + \Delta H_{2} + \Delta H_{3} + \ldots + \Delta H_{n}\]

\newpage

\subsection{Haber process}
\label{sec:org4768217}
The Haber process is just the industrial process to create ammonia (\(NH_3\)) from \(H_2\) and \(N_2\) to form ammonia (\(NH_3\)).
\\[0pt]

The conditions are:
\begin{enumerate}
\item Pressure of \(\qty{200}{\unit{atm}}\)
\item Temperature of \(\qty{450}{\unit{\degreeCelsius}}\)
\item Presence of iron (\(Fe\)) catalyst
\end{enumerate}

\[3H_2 (g) + N_2 (g) \rightarrow 2NH_3 (g) \qquad \Delta H^\circ = -\qty{92.2}{\unit{kJ}}\]

\subsection{Standard enthalpy change of formation (\(\Delta H^\circ_f\)) (Standard heat of formation)}
\label{sec:org7e2580c}
The standard enthalpy change of formation is the enthalpy change for the formation of \(\qty{1}{\unit{mol}}\) of a substance in its \textbf{standard state} from its constituent elements in their \textbf{standard states}.

\subsection{Bond dissociation energies}
\label{sec:orge197ed3}
Bond dissociation energies is the \textbf{standard enthalpy changes} for the corresponding \textbf{bond-breaking} reactions.
\\[0pt]

Essentially, it is \(\Delta H^\circ_{Bond-breaking}\).

\subsection{Spontaneous process}
\label{sec:org95e60e2}
A spontaneous process is a process that, once started, proceeds on its own \textbf{without} a continuous \textbf{external influence}.

\subsection{Entropy (\(S\))}
\label{sec:org21d652e}
Entropy is the amount of molecular randomness in a system.

\subsection{Change in Gibbs Free Energy (\(\Delta G\))}
\label{sec:org3ed1fb2}
\[\Delta G = \Delta H - T \Delta S\]

\begin{itemize}
\item When \(\Delta G < 0\), the reaction is \textbf{spontaneous}.
\item When \(\Delta G = 0\), the reaction is at \textbf{equilibrium}.
\item When \(\Delta G > 0\), the reaction is \textbf{not spontaneous}.
\end{itemize}

\newpage

\section{Formulas}
\label{sec:org6a77c77}

\subsection{Change in internal energy due to change in pressure or volume}
\label{sec:org1d6d609}
\[q = \text{Heat transferred}\]
\[\text{Work done: } w = - P \Delta V\]

\[\Delta E = q + w\]
\[q = \Delta E + P \Delta V\]

For constant volume (\(\Delta V = 0\)):
\[q_v = \Delta E\]

For constant pressure:
\[q_p = \Delta E + P \Delta V\]

Since enthalpy change (\(\Delta H\)) is equal to the heat transferred:
\[\Delta H = q_p = \Delta E + P \Delta V\]

Since enthalpy is a state function whose value depends only on the current state of the system:
\begin{align*}
\Delta H &= H_{final} - H_{initial} \\
&= H_{products} - H_{reactants} \\
&= \Delta (H_f)_{products} - \Delta (H_f)_{reactants} \\
&= \Delta H_{Bond-breaking} - \Delta H_{Bond-forming}
\end{align*}
\end{document}