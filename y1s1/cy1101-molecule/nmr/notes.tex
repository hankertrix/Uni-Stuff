% Created 2023-08-29 Tue 01:43
% Intended LaTeX compiler: pdflatex
\documentclass[11pt]{article}
\usepackage[utf8]{inputenc}
\usepackage[T1]{fontenc}
\usepackage{graphicx}
\usepackage{longtable}
\usepackage{wrapfig}
\usepackage{rotating}
\usepackage[normalem]{ulem}
\usepackage{amsmath}
\usepackage{amssymb}
\usepackage{capt-of}
\usepackage{hyperref}
\usepackage{chemfig, siunitx}
\author{Hankertrix}
\date{\today}
\title{NMR Lecture Notes}
\hypersetup{
 pdfauthor={Hankertrix},
 pdftitle={NMR Lecture Notes},
 pdfkeywords={},
 pdfsubject={},
 pdfcreator={Emacs 29.1 (Org mode 9.6.6)}, 
 pdflang={English}}
\begin{document}

\maketitle
\setcounter{tocdepth}{2}
\tableofcontents

\newpage

\section{NMR theory}
\label{sec:orgd934d67}
A nucleus with an odd number of protons or neutrons will have an intrinsic nuclear spin. A few examples include Hydrogen and Fluorine and Carbon 13. For this module, we will only be looking at one-dimensional Hydrogen (or proton) NMR.

When there is no magnetic field applied to a nucleus, the nuclear spin is random. The nuclear spin in this state is called degenerate.

However, when a nucleus with a non-zero spin is placed in a magnetic field, the nuclei can either be aligned in the same direction, or in the opposite direction of the magnetic field. This lifts the degeneracy of the nucleus. These two nuclear spin alignments have different energies, with the nucleus that is aligned to the magnetic field having lower energy than a nucleus that is aligned in the opposite direction to the field.

Nuclear magnetic resonance (NMR) spectroscopy is based on the absorption of radio frequency (RF) radiation by the nucleus in a strong magnetic field. Absorbing a \textbf{specific} radio frequency causes the nuclear spin to realign, leaving it in a higher energy state. After the nucleus absorbs the energy, it will re-emit the absorbed energy as RF radiation and return to a lower energy state.

\subsection{How does this help with NMR?}
\label{sec:orga30e70b}
\begin{itemize}
\item The resonating frequency of the energy transition is dependent on the effective magnetic field at the nucleus.
\item This field is affected by electron shielding, which is dependent on the chemical environment around the nucleus.
\item Hence, this dependency of the resonating frequency on the chemical environment of a particular atom in a molecule allows NMR spectroscopy to determine the structure of molecules.
\end{itemize}

\newpage

\subsection{Electron shielding}
\label{sec:org02ed426}

The electrons around the nucleus are induced by the external magnetic field to spin, which creates a tiny current in the nucleus. This will cause the nucleus to set up its own small magnetic field that opposes the applied magnetic field.

This magnetic field will shield the underlying nucleus from the full effect of the applied magnetic field by a small but measurable amount. Hence, the effective magnetic field will not be the same as the applied magnetic field.

Thus, the electron density around the nucleus can be used as a tool to differentiate the signals generated by the different nuclei in a molecule that have different electron densities around them.

\subsubsection{Effect of electron shielding}
\label{sec:org988618d}
With greater electron density around the nucleus.
\[\downarrow\]
The nucleus is shielded from the magnetic field to a greater extent.
\[\downarrow\]
With greater shielding, the effective magnetic field felt by the nucleus is weaker.
\[\downarrow\]
The weaker magnetic field causes the energy gap between the two nuclear spin states to be smaller.
\[\downarrow\]
The smaller energy gap means that less energy is required to flip the nuclei from a lower energy spin state to a higher energy spin state.
\[\downarrow\]
The energy is proportional to the frequency, so a nucleus that is shielded by its greater electron density will have a lower resonance frequency compared to a nucleus that has lower electron density.

\newpage

\section{Fourier transform}
\label{sec:orgd8fcb38}

\subsection{Signal processing}
\label{sec:org79592d3}
The frequencies are extracted from the Free Induction Decay (FID) by a Fourier transform of the time-based data, where the x-axis is the time and the y-axis is the frequency. The result of the Fourier transform is the NMR resonance signal, where the time domain is transformed into frequency.

The scale is made more manageable by expressing it in parts per million (ppm) and is independent of the spectrometer operating frequency.

\begin{align*}
\text{Chemical Shift, } \delta &= \frac{\text{Frequency of signal - Frequency of reference}}{\text{Spectrometer frequency}} \times 10^6 \\
\delta &= \frac{\text{Observed chemical shift (Number of \si{\hertz} away from TMS)}}{\text{Spectrometer frequency in \si{\mega\hertz}}}
\end{align*}

\section{Field lock and reference}
\label{sec:org40c88ea}

\subsection{Why do we use deuterated solvents in sample preparation?}
\label{sec:orgaded52a}
A routine sample is 1D-1H-NMR and approximately 5 mg - 20 mg of the compound is added to about 0.4ml of solvent in a 5 mm o.d. glass tube. It is important that the sample doesn't produce confusing or complicated signals in the spectrum. Hence, deuterium atoms are used as they have sufficiently different magnetic properties from ordinary hydrogen and don't produce signals in the area of the spectrum that we are looking at in 1H NMR. The solvents that have their hydrogen atoms replaced by deuterium, like \(CDCl_3\) (instead of \(CHCl_3\)) are called deuterated solvents. Usually, we will use 99.8\% deuterated solvent, which means that 99.8\% of the hydrogen atoms in the solvent are replaced with deuterium.

\newpage

\subsubsection{Field lock}
\label{sec:org9ae8083}
Another reason to use deuterated solvents is because the magnetic field strength might vary over time due to various factors, such as the ageing of the magnet, the movement of metal objects near the magnet, and temperature fluctuations. The spectrometer will constantly monitor the resonance frequency of the deuterated solvent and makes minor changes to the magnetic field to keep the resonance frequency of the deuterated solvent constant. Hence, the deuterated solvent is used as a reference and there is field lock.
\\[0pt]

The molecule called Tetramethylsilane (TMS) is often used as a reference as it is a very shielded compound which always gives a resonance signal of 0ppm. TMS is also very unreactive and doesn't react with most other compounds. As such, it is used as a reference to calibrate the scale for NMR spectroscopy.

\[
\chemname{
\chemfig{Si(-[:0]CH_3)(-[:90]CH_3)(-[:180]CH_3)(-[:270]CH_3)}
}{Tetramethylsilane (TMS)}
\]


\section{Chemical shift}
\label{sec:orgb801336}
An NMR spectrum is a plot of the radio frequency applied (x-axis) against absorption (y-axis). A signal in the spectrum is called resonance, \textbf{not peak}. The frequency of the signal is known as its chemical shift. So, the chemical shift is defined by the frequency of the resonance expressed with reference to a standard compound which is defined to be at 0 ppm. A \textbf{shielded} proton is said to be \textbf{upfield} while a \textbf{deshielded} proton is said to be \textbf{downfield} in a NMR spectroscopy.

The magnetic field experienced by a proton is influenced by various structural factors. These include:
\begin{itemize}
\item Inductive effects by electronegative groups
\item Magnetic anisotropy
\item Hydrogen bonding
\end{itemize}

\newpage

\subsection{Inductive effects by electronegative groups}
\label{sec:orgfd98e73}
Since the electrons around the proton creates a magnetic field (\(H_e\)) that opposes the applied magnetic field (\(H_0\)), this reduces the effective magnetic field (\(H_{eff}\)) experienced by the proton. Hence, the electrons are said to shield the protons. As the field experienced by the proton (\(H_{eff}\)) defines the energy difference between the two spin states, the frequency and hence the chemical shift, in ppm, will change depending on the electron density around the proton.

Electronegative groups such as \(-OH\), \(-F\) and \(-Cl\) that are attached to the \(C-H\) system decrease the electron density around the protons due the electronegative groups pulling the electrons away from the protons. Therefore, there is less shielding and the chemical shift increases.

The effects are cumulative, so the presence of more electronegative groups will cause a greater increase in the chemical shift.

These inductive effects are not just felt by the immediately adjacent protons but are also felt in the atoms further down the chain. The effect does however fade away rapidly as you move away from the electronegative group.

\newpage

\subsection{Magnetic anisotropy}
\label{sec:orgfec7031}
Electrons in \(\pi\) systems interact with the applied magnetic field which induces a magnetic field that causes anisotropy. As a result, the nearby protons will experience 3 fields, which are:
\begin{itemize}
\item The applied field (\(H_0\))
\item The shielding field (\(H_e\))
\item The field due to the \(\pi\) system
\end{itemize}

Depending on the position of this third field, the proton can either be shielded (smaller \(\delta\)) or deshielded (larger \(\delta\)). In the case of aryl hydrogen atoms (the hydrogen atoms attached to a benzene ring), the \(H_{eff}\) is larger than the \(H_0\), which means that the proton is deshielded. In the case of vinylic hydrogen atoms (hydrogen atoms attached to an alkene group), the protons are also deshielded, but to a smaller extent compared to the aryl hydrogen atoms. For most other cases that we will encounter, the proton will be deshielded as well.

\newpage

\subsubsection{How does magnetic anisotropy work?}
\label{sec:org5c9919a}
When a magnetic field is applied to a molecule that has a \(\pi\) system (double or triple bond, or a benzene ring), the applied magnetic field will cause the electrons in the \(\pi\) system to spin. This in turn causes the molecule to set up its own magnetic field, called an \textbf{induced magnetic field}, at the \textbf{centre} of the \(\pi\) system, which is at the centre of the double or triple bond for non-aromatic compounds, and at the centre of the benzene ring for aromatic compounds. The induced magnetic field that is set up by the \(\pi\) system will always \textbf{oppose} the applied magnetic field. Hence, the induced magnetic field will always \textbf{oppose} the applied magnetic field at the \textbf{centre} of the \(\pi\) system.
\\[0pt]

To determine if the effective magnetic field \(H_{eff}\) will be greater or smaller than the applied magnetic field \(H_{0}\), we will need to draw the magnetic field lines going out in a circle around the \textbf{centre} of the \(\pi\) system, where the induced magnetic field is set up. The direction of the magnetic field lines originating from the \textbf{induced magnetic field} at the hydrogen atom of interest will tell you if the \(H_{eff}\) will be greater or smaller than the applied magnetic field \(H_{0}\).
\\[0pt]

If the direction of the magnetic field lines originating from the induced magnetic field is in the \textbf{same} direction as the applied magnetic field at the hydrogen atom of interest, then the effective magnetic field \(H_{eff}\) will be \textbf{greater} than the applied magnetic field \(H_{0}\).
\\[0pt]

If the direction of the magnetic field lines originating from the induced magnetic field is in the \textbf{opposite} direction of the applied magnetic field at the hydrogen atom of interest, then the effective magnetic field \(H_{eff}\) will be \textbf{smaller} than the applied magnetic field \(H_{0}\).
\\[0pt]

Generally, the hydrogen atoms that are located \textbf{in} the centre (or \textbf{close} to the centre) of a \(\pi\) system will experience a \textbf{smaller} effective magnetic field \(H_{eff}\) while the hydrogen atoms that are located \textbf{away} from the centre of a \(\pi\) system (which means they are \textbf{towards the side} and not in the middle) will experience a \textbf{greater} effective magnetic field \(H_{eff}\). Do note that the hydrogen atoms must be directly bonded to the carbon atom with the double or triple bond to be affected by magnetic anisotropy.
\\[0pt]

Here is a \href{https://youtu.be/w8ew5bvdrqg}{video} to watch if you still have trouble understanding the concept.

\newpage

\subsection{Hydrogen bonding}
\label{sec:org7f081d8}
Protons that are involved in hydrogen bonding (usually \(-OH\) and \(-NH\)) are typically observed \textbf{over a large range of chemical shift values}. The more hydrogen bonding there is, the more the proton is deshielded and the higher its chemical shift will be.

\subsubsection{Determining between \(-OH\) and \(-NH\)}
\label{sec:org394b98c}
\(-OH\) and \(-NH\) protons can be identified by carrying out a simple \(D_2O\) (heavy water) exchange experiment.
\begin{enumerate}
\item Run the regular H-NMR experiment
\item Add a few drops of \(D_2O\)
\item Re-run the H-NMR experiment
\item Compare the two spectra and look for signals that have "disappeared"
\end{enumerate}

\subsubsection{Why would a signal disappear?}
\label{sec:org0f1564e}
Consider the alcohol case for example:
\[R-OH + D_2O \rightleftharpoons R-OD + HOD\]

During the hydrogen bonding, the alcohol and heavy water can "exchange" -H and -D amongst each other, so the alcohol becomes \(R-OD\). Although \(D\) is NMR active, its signals are of different energy and will not be seen in the H-NMR. Thus, the signal due to the \(-OH\) disappears. The signal due to HOD will appear, however.

\newpage

\section{Spin-spin coupling}
\label{sec:org2fb9722}
Consider the H-NMR of ethanol, \(CH_3 - CH_2 - OH\). The methyl group refers to the \(-CH_3\) group and the methylene group refers to the \(-CH_2-\) group. There are two kinds of signals that can be expected as the methyl protons and the methylene protons have different electronic environments. The signals created as a result of spin-spin coupling is due to the hydrogen atoms not being excited at the same time, which means that there are still some unexcited hydrogen atoms that will create the signals associated with spin-spin coupling.
\\[0pt]

There are a good deal of different multiplicity of signals, such as:
\begin{enumerate}
\item Singlet
\item Doublet
\item Triplet
\item Quartet
\item Quintet
\end{enumerate}

And so on\ldots{}

\subsection{Definition of magnetic equivalence}
\label{sec:orgd59d887}
Nuclei having the same resonance frequency in nuclear magnetic resonance spectroscopy and also identical spin-spin interactions with the nuclei of a neighbouring group are \textbf{magnetically equivalent}. The spin-spin interaction between magnetically equivalent nuclei does not appear, and thus has no effect on the multiplicity of the respective NMR signals.

\newpage

\subsection{Will the 3 protons in the methyl (\(-CH_3\)) group show as different signals?}
\label{sec:orgcbab92a}
Yes. The signal from the methyl group has been split into 3 different signals, also known as a triplet. The signal from the methylene group \(-CH_2-\) has been split into 4 different signals, also known as a quartet. This occurs due to a small interaction between two groups of protons, which is known as spin-spin coupling.
\\[0pt]

The reason why hydrogen atom that is bonded to the oxygen atom in the \(-OH\) group only shows as a single signal is because the oxygen is highly electronegative, which blocks the signal from spin-spin coupling from getting to the hydrogen atom.

In general, an electronegative group in the way of a hydrogen atom will block the signals from spin-spin coupling, and hence the hydrogen atom will only show a single signal that has not been split.

\subsubsection{Why is the methyl (\(-CH_3\)) group split into a triplet?}
\label{sec:orgb882a3f}
We will have to look at the protons in the adjacent methylene group (\(-CH_2-\)). There are 2 protons in the methylene group (\(-CH_2-\)) and each one can have one of two possible nuclear spin orientations, either aligned with the magnetic field or opposed to the direction of the magnetic field. Here is the list of possible spin states:
\begin{itemize}
\item \(\rightrightarrows\)
\item \(\leftleftarrows\)
\item \(\rightleftarrows\)
\item \(\leftrightarrows\)
\end{itemize}

The last two spin states are considered the same, so there are 3 possible spin states with the ratio:
\begin{align*}
\rightrightarrows : \rightleftarrows : \leftleftarrows \\ 1 : 2 : 1
\end{align*}

Hence, the methyl (\(-CH_3\)) group is split into a triplet.

\newpage

\subsubsection{Why is the methylene (\(-CH_2-\)) group split into a quartet?}
\label{sec:orgd02caaa}
Similarly, we will have to look at the protons in the adjacent methyl (\(-CH_3\)) group. There are 3 protons in the methyl (\(-CH_3\)) group, so here's the list of possible spin states:
\\[0pt]

All 3 spin states aligned with the magnetic field.
\begin{itemize}
\item \(\mathrel{\substack{\textstyle\leftarrow \\[-0.6ex] \textstyle\leftarrow \\[-0.6ex] \textstyle\leftarrow}}\)
\end{itemize}

All 3 spin states opposed to the direction of the magnetic field.
\begin{itemize}
\item \(\mathrel{\substack{\textstyle\rightarrow \\[-0.6ex] \textstyle\rightarrow \\[-0.6ex] \textstyle\rightarrow}}\)
\end{itemize}

2 spin states aligned with the magnetic field and 1 spin state opposed to the direction of the magnetic field. These spin states are all considered the same.
\begin{itemize}
\item \(\mathrel{\substack{\textstyle\leftarrow \\[-0.6ex] \textstyle\leftarrow \\[-0.6ex] \textstyle\rightarrow}}\)
\item \(\mathrel{\substack{\textstyle\leftarrow \\[-0.6ex] \textstyle\rightarrow \\[-0.6ex] \textstyle\leftarrow}}\)
\item \(\mathrel{\substack{\textstyle\rightarrow \\[-0.6ex] \textstyle\leftarrow \\[-0.6ex] \textstyle\leftarrow}}\)
\end{itemize}

2 spin states opposed to the direction of the magnetic field and 1 spin state aligned with the magnetic field. These spin states are all considered the same.
\begin{itemize}
\item \(\mathrel{\substack{\textstyle\rightarrow \\[-0.6ex] \textstyle\rightarrow \\[-0.6ex] \textstyle\leftarrow}}\)
\item \(\mathrel{\substack{\textstyle\rightarrow \\[-0.6ex] \textstyle\leftarrow \\[-0.6ex] \textstyle\rightarrow}}\)
\item \(\mathrel{\substack{\textstyle\leftarrow \\[-0.6ex] \textstyle\rightarrow \\[-0.6ex] \textstyle\rightarrow}}\)
\end{itemize}

Hence, there are 4 possible spin states with the ratio:
\begin{align*}
\mathrel{\substack{\textstyle\leftarrow \\[-0.6ex] \textstyle\leftarrow \\[-0.6ex] \textstyle\leftarrow}} : \mathrel{\substack{\textstyle\leftarrow \\[-0.6ex] \textstyle\leftarrow \\[-0.6ex] \textstyle\rightarrow}} : \mathrel{\substack{\textstyle\rightarrow \\[-0.6ex] \textstyle\rightarrow \\[-0.6ex] \textstyle\leftarrow}} : \mathrel{\substack{\textstyle\rightarrow \\[-0.6ex] \textstyle\rightarrow \\[-0.6ex] \textstyle\rightarrow}} \\ 1 : 3 : 3 : 1
\end{align*}

Thus, the methylene (\(-CH_2-\)) group is split into a quartet.

\subsection{The (n+1) rule}
\label{sec:org0747e79}
The multiplicity of a multiplet (doublet, triplet, etc.) is given by the number of equivalent protons in the neighbouring atoms plus one. Hence, the n in the (n+1) rule is the number of equivalent protons in the neighbouring atoms.

Equivalent nuclei do not interact with each other. The three methyl protons in ethanol causes splitting of the neighbouring methylene protons, but they \textbf{do not cause splitting amongst themselves}.

\subsection{Pascal's Triangle}
\label{sec:orgaee5b57}
The relative intensities of the lines in a group of split signals is given by a binomial expansion, or Pascal's Triangle. So for a 1H-NMR, a proton with 0 neighbours, appears as a single line, while a proton with 1 neighbour will have 2 lines of equal intensity, and a proton with 2 neighbours will have 3 lines of intensities in the ratio of 1:2:1.

\subsection{Identifying multiplets}
\label{sec:org0eb0ad5}
Multiplets have the same separation between the signals, which is called a coupling constant (\(J\)). This allows them to be easily distinguished from closely spaced chemical shift signals. For ethanol (\(CH_3 - CH_2 - OH\)), the spacing between the signals of the methyl (\(-CH_3\)) triplet are equal to the spacing between the signals of the methylene (\(-CH_2-\)) quartet because they couple to each other, and \textbf{only to each other}.
\\[0pt]

This spacing is measured in \(\si{\hertz}\) and is called the coupling constant (\(J\)).

\subsection{How to calculate the coupling constant (\(J\))}
\label{sec:orgc91c1fb}
\[\text{Operating frequency } (\si{\mega\hertz}) \times \text{Splitting } (\si{ppm}) = \text{Coupling Constant } (\si{\hertz})\]

Splitting is the difference between the \(\si{ppm}\) values of each signal of the multiplet. It is the gap between the individual signals of the multiplet. For a multiplet that has more than one gap, such as a quartet, you should take the average of all the gaps to calculate the coupling constant. The final (\(J\)) value is usually rounded to \textbf{one decimal place}.

Coupled protons will have the same \(J\) values if they don't have any other couplings.

\newpage

\subsection{What causes the signal to split?}
\label{sec:org9089750}
A nucleus under examination is perturbed by a nearby nuclear spin.
\[\downarrow\]
The currently observed nucleus will respond the perturbation, which is shown in its resonance signal.
\[\downarrow\]
This spin coupling is \textbf{transmitted through the connecting bonds}.
\[\downarrow\]
When the perturbing nucleus becomes the observed nucleus, it also exhibits the same signal splitting with the same coupling constant (\(J\)).
\\[0pt]

For spin coupling to be observed, the interacting nuclei must be bonded in relatively close proximity (\textbf{usually about 3 bonds away}), or be oriented in certain optimal and rigid configurations. Spectroscopists place a number before the symbol \(J\) to designate the number of bonds linking the coupled nuclei (shown in orange below). Using this terminology, a geminal coupling constant is \(^2J\) and a vicinal coupling constant is \(^3J\).

\[\chemfig{C(<[:45,,,,orange]{\color{blue}H})(<:[:135,,,,orange]{\color{blue}H})(-[:315]R)(-[:225]R)}\]
\[\text{Geminal Hydrogens (Hydrogen atoms that are bonded to the same atom)}\]

\[
\chemfig{C(-[:90,,,,orange]{\color{blue}H})(-[:180]R)(-[:270]R)
(-[,,,,orange]C(-R)(-[:90,,,,orange]{\color{blue}H})(-[:270]R))}
\]
\[\text{Vicinal Hydrogens (Hydrogen atoms on adjacent atoms)}\]
\end{document}