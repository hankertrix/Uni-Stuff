% Created 2023-09-25 Mon 02:13
% Intended LaTeX compiler: pdflatex
\documentclass[11pt]{article}
\usepackage[utf8]{inputenc}
\usepackage[T1]{fontenc}
\usepackage{graphicx}
\usepackage{longtable}
\usepackage{wrapfig}
\usepackage{rotating}
\usepackage[normalem]{ulem}
\usepackage{amsmath}
\usepackage{amssymb}
\usepackage{capt-of}
\usepackage{hyperref}
\usepackage{pgfplots, siunitx}
\author{Hankertrix}
\date{\today}
\title{Math Module 3A Tutorial}
\hypersetup{
 pdfauthor={Hankertrix},
 pdftitle={Math Module 3A Tutorial},
 pdfkeywords={},
 pdfsubject={},
 pdfcreator={Emacs 29.1 (Org mode 9.6.6)}, 
 pdflang={English}}
\begin{document}

\maketitle
\setcounter{tocdepth}{2}
\tableofcontents

\newpage

\section{Question 1}
\label{sec:org4a02a0d}

Let \(f(x) = 2\sqrt{x} - (3 - \frac{1}{x})\).
\[f(x) = 2\sqrt{x} - 3 + \frac{1}{x}\]

Finding \(f'(x)\):
\begin{align*}
f'(x) &= \frac{2}{\sqrt{x}} \cdot \frac{1}{2} - \frac{1}{x^2} \\
&= \frac{1}{\sqrt{x}} - \frac{1}{x^2}
\end{align*}

At \(x = 1\):
\begin{align*}
f(1) &= 2 \sqrt{1} - 3 + \frac{1}{1} \\
&= 0
\end{align*}

\begin{align*}
f'(1) &= \frac{1}{\sqrt{1}} - \frac{1}{1^2} \\
&= 0
\end{align*}

This means that \(x = 1\) is a stationary point.

At \(x = 1^+\):
\begin{align*}
f'(1^+) &= \frac{1}{\sqrt{1^+} - \frac{1}{(1^+)^2}} \\
&> 0
\end{align*}

Thus, when \(x > 1\):
\[f(x) > 0\]
\[2 \sqrt{x} - (3 - \frac{1}{x}) > 0\]
\[2 \sqrt{x} > 3 - \frac{1}{x} \textbf{ (Proven)}\]


\section{Question 2}
\label{sec:org96c1805}

\[|\arctan x - \arctan y| \le | x - y |\]
\[\frac{|\arctan x - \arctan y|}{| x - y |} \le 1\]

Let \(f(x)\) be \(\arctan x\):
\[f'(x) = \frac{1}{1 + x^2}\]

At an arbitrary point \(c\):
\[\left|\frac{\arctan x - \arctan y}{x - y}\right| = |f'(c)|\]
\[\frac{|\arctan x - \arctan y|}{|x - y|} = \left|\frac{1}{1 + c^2}\right|\]

The maximum value of \(\left| \frac{1}{1 + c^2}\right|\) is 1, when \(c = 0\). When \(c > 0\), \(\left| \frac{1}{1 + c^2} \right| < 1\).

Hence:
\[\frac{|\arctan x - \arctan y|}{|x - y|} \le 1 \textbf{ (Proven)}\]

\newpage

\section{Question 3}
\label{sec:orga19c326}

\subsection{(a)}
\label{sec:org6ffc018}

Proving the base cases:
\\[0pt]

When \(r = 0\):
\[(fg)^{(0)} = \sum_{k = 0}^{0} {0 \choose k} f^{(k)} g^{(0 - k)}\]
\[fg = {0 \choose 0} f^{(0)} g^{(0)}\]
\[fg = 1 f g\]
\[fg = fg \textbf{ (Proven)}\]

When \(r = 1\):
\[(fg)^{(1)} = \sum_{k = 0}^{1} {1 \choose k} f^{(k)} g^{(1 - k)}\]
\[f'g + fg' = {1 \choose 0} f^{(0)} g^{(1 - 0)} + {1 \choose 1} f^{(1)} g^{(1 - 1)}\]
\[f'g + fg' = fg' + f'g\]
\[f'g + fg' = f'g + fg' \textbf{ (Proven)}\]

Assuming the equation holds for all \(n \in \mathbb{R}\):
\begin{align*}
(fg)^{(n + 1)} &= \left(\sum_{k = 0}^{n} {n \choose k} f^{(k)} g^{(n - k)}\right)' \\
&= \sum_{k = 0}^{n} {n \choose k} f^{(k + 1)} g^{(n - k)} + \sum_{k = 0}^{n} {n \choose k} f^{(k)} g^{(n - k + 1)} \\
&= \sum_{k' = 1}^{n} {n \choose k' - 1} f^{(k')} g^{(n - (k' - 1))} + \sum_{k = 0}^{n} {n \choose k} f^{(k)} g^{(n + 1 - k)} \\
&= \sum_{k' = 1}^{n} {n \choose k' - 1} f^{(k')} g^{(n - k' + 1)} + \sum_{k = 0}^{n} {n \choose k} f^{(k)} g^{(n + 1 - k)} \\
&= \sum_{k' = 1}^{n} {n \choose k' - 1} f^{(k')} g^{(n + 1 - k')} + \sum_{k = 0}^{n} {n \choose k} f^{(k)} g^{(n + 1 - k)} \\
&= \sum_{k' = 1}^{n} {n \choose k' - 1} f^{(k')} g^{(n + 1 - k')} + \sum_{k = 1}^{n} {n \choose k} f^{(k)} g^{(n + 1 - k)} + {n \choose 0} f^{(0)} g^{(n + 1 - 0)} \\
&= \sum_{k' = 1}^{n} {n \choose k' - 1} f^{(k')} g^{(n + 1 - k')} + \sum_{k = 1}^{n} {n \choose k} f^{(k)} g^{(n + 1 - k)} + f g^{(n + 1)} \\
&= \sum_{k = 1}^{n} \left( {n \choose k - 1} + {n \choose k} \right) f^{(k)} g^{(n + 1 - k)} + f g^{(n + 1)} \\
&= \sum_{k = 1}^{n + 1} {n + 1 \choose k} f^{(k)} g^{(n + 1 - k)} + f g^{(n + 1)} \\
&= \sum_{k = 0}^{n + 1} {n + 1 \choose k} f^{(k)} g^{(n + 1 - k)} \quad (\because fg^{(n + 1)} = f^{(k)} g^{(n + 1 - k)} \text{ when } k = 0) \ \textbf{ (Proven)} \\
\end{align*}

\subsection{(b)}
\label{sec:org9644bd9}

\begin{align*}
{n \choose k - 1} + {n \choose k} &= \frac{n!}{(k - 1)!(n - (k - 1))!} + \frac{n!}{k!(n - k)!} \\
&= \frac{n!}{(k - 1)!(n - k + 1)!} + \frac{n!}{k!(n - k)!} \\
&= \frac{kn!}{k(k - 1)!(n - k + 1)!} + \frac{n!(n - k + 1)}{k!(n - k + 1)(n - k)!} \\
&= \frac{kn!}{k!(n - k + 1)!} + \frac{n!(n - k + 1)}{k!(n - k + 1)!} \\
&= \frac{n!(k + n - k + 1)}{k!(n - k + 1)!} \\
&= \frac{n!(n + 1)}{k!(n + 1 - k)!} \\
&= \frac{(n + 1)!}{k!(n + 1 - k)!} \\
&= {n + 1 \choose k} \quad \textbf{ (Proven)}
\end{align*}

\newpage

\section{Question 4}
\label{sec:org04fc90b}

\[
g(x) = \begin{cases}
x + \sin 2x, & \text{for } x \in (0, \frac{\pi}{2}) \\
x - \sin 2x, & \text{for } x \in \left(\frac{\pi}{2}, \pi \right)
\end{cases}
\]

Differentiating \(g\) with respect to \(x\):
\[
g'(x) = \begin{cases}
1 + 2\cos 2x, & \text{for } x \in (0, \frac{\pi}{2}) \\
1 - 2\cos 2x, & \text{for } x \in \left(\frac{\pi}{2}, \pi \right)
\end{cases}
\]

Finding the stationary points:
\[g'(x) = 0\]
\begin{align*}
1 + 2\cos 2x = 0 && 1 - 2\cos 2x &= 0 \\
2\cos 2x = -1 && - 2\cos 2x &= -1 \\
2\cos 2x = -1 && 2\cos 2x &= 1 \\
\cos 2x = -\frac{1}{2} && \cos 2x &= \frac{1}{2} \\
2x = \frac{2\pi}{3} && 2x &= 2\pi - \frac{\pi}{3} \quad \left(\because x \in \left(\frac{\pi}{2}, \pi \right) \right) \\
x = \frac{\pi}{3} && 2x &= \frac{5}{3} \pi \\
&& x &= \frac{5\pi}{6} \\
\end{align*}

\begin{center}
\begin{tabular}{|c|c|c|c|}
\hline
$x$ & \(\left( \frac{\pi}{3} \right)^-\) & \(\frac{\pi}{3}\) & \(\left( \frac{\pi}{3} \right)^+\) \\
\hline
$g'(x)$ & $0^+$ & 0 & $0^-$ \\
\hline
Shape & $\slash$ & $-$ & $\backslash$ \\
\hline
\end{tabular}
\hskip 3em
\begin{tabular}{|c|c|c|c|}
\hline
$x$ & \(\left( \frac{5\pi}{6} \right)^-\) & \(\frac{5\pi}{6}\) & \(\left( \frac{5\pi}{6} \right)^+\) \\
\hline
$g'(x)$ & $0^+$ & 0 & $0^-$ \\
\hline
Shape & $\slash$ & $-$ & $\backslash$ \\
\hline
\end{tabular}
\end{center}

Hence, both \(x = \frac{\pi}{3}\) and \(x = \frac{5\pi}{6}\) are maximum points.
\\[0pt]

Since \(f\) is \(g\) repeated over the entire domain of \(x \in \mathbb{R}\), the maximum points will be:

\[x = \frac{\pi}{3} + k \pi \text{ and } x = \frac{5\pi}{6} + k \pi, \text{ where } k \in \mathbb{Z}\]

Thus, the part of the graph that will be increasing are in the intervals \((0 + k\pi, \frac{\pi}{3} + k\pi)\) and \((\frac{\pi}{2} + k \pi, \frac{5\pi}{6} + k\pi)\), where \(k \in \mathbb{Z}\). The part of the graph that will be decreasing are in the intervals \((\frac{\pi}{3} + k\pi, \frac{\pi}{2} + k \pi)\) and \(\frac{5\pi}{6} + k\pi, \pi + k\pi\), where \(k \in \mathbb{Z}\).


\section{Question 5}
\label{sec:orga36184f}

\[f(x) = \frac{x^2}{x^2 + 3}\]

Differentiating \(f\) with respect to \(x\):
\begin{align*}
f'(x) &= \frac{(x^2 + 3) \cdot 2x - x^2 \cdot 2x}{(x^2 + 3)^2} \\
&= \frac{2x^3 + 6x - 2x^3}{(x^2 + 3)^2} \\
&= \frac{6x}{(x^2 + 3)^2}
\end{align*}

Finding all the stationary points of \(f\):
\[f'(x) = 0\]
\[\frac{6x}{(x^2 + 3)^2} = 0\]
\[6x = 0\]
\[x = 0\]

Hence, \(x = 0\) is a stationary point. At \(x = 0\):

\begin{center}
\begin{tabular}{|c|c|c|c|}
\hline
$x$ & \(0^-\) & \(0\) & \(0^+\) \\
\hline
$f'(x)$ & $0^-$ & 0 & $0^+$ \\
\hline
Shape & $\backslash$ & $-$ & $\slash$ \\
\hline
\end{tabular}
\end{center}

Thus, \(x = 0\) is a local and global minimum.
\\[0pt]

Since \(x = 0\) is a global minimum, \(f\) is strictly decreasing in the interval \((- \infty, 0)\) and strictly increasing in the interval \((0, \infty)\).

Differentiating \(f'(x)\) with respect to \(x\):
\begin{align*}
f''(x) &= \frac{6(x^2 + 3)^2 - 2(x^2 + 3) \cdot 2x \cdot 6x}{((x^2 + 3)^2)^2} \\
&= \frac{6(x^2 + 3)^2 - 24x^2(x^2 + 3)}{(x^2 + 3)^4} \\
&= \frac{(x^2 + 3) (6(x^2 + 3) - 24x^2)}{(x^2 + 3)^4} \\
&= \frac{6(x^2 + 3) - 24x^2}{(x^2 + 3)^3} \\
&= \frac{6x^2 + 18 - 24x^2}{(x^2 + 3)^3} \\
&= \frac{6x^2 - 24x^2 + 18}{(x^2 + 3)^3} \\
&= \frac{-18x^2 + 18}{(x^2 + 3)^3} \\
&= \frac{18(1 - x^2)}{(x^2 + 3)^3} \\
\end{align*}

Since \((x^2 + 3)^3 > 0\) for \(x \in \mathbb{R}\), we can look at the numerator of \(f''(x)\):
\\[0pt]

\(18(1 - x^2) > 0\) when \(x^2 < 1\), which is when \(-1 < x < 1\). Thus, \(f\) is concave up from \((-1, 1)\) and \(f\) is concave down elsewhere, which is on the intervals \((-\infty, -1)\) and \((1, \infty)\).

\begin{center}
\begin{tikzpicture}
\begin{axis}[axis lines = center, ymin = -1, ymax = 2]
\addplot[color = blue, samples = 500, domain = -10:10]{(x^2)/(x^2 + 3)};
\end{axis}
\end{tikzpicture}
\end{center}


\section{Question 6}
\label{sec:org1033de6}

Let \(\theta\) be the angle of inclination of the ladder and \(L(\theta)\) be the length of the ladder.
\[L(\theta) = \frac{1}{\cos \theta} + \frac{2}{\sin \theta}, \quad \theta \in \left(0, \frac{\pi}{2} \right)\]

Differentiating \(L(\theta)\) with respect to \(\theta\):
\begin{align*}
L'(\theta) &= \frac{1}{\cos^2 \theta} \cdot \sin \theta + \frac{2}{\sin^2 \theta} \cdot - \cos \theta \\
&= \frac{\sin \theta}{\cos^2 \theta} - \frac{2\cos \theta}{\sin^2 \theta}
\end{align*}

Setting \(L'(\theta)\) to \(0\) to find the stationary points:
\[L'(\theta) = 0\]
\[\frac{\sin \theta}{\cos^2 \theta} - \frac{2 \cos \theta}{\sin^2 \theta} = 0\]
\[\frac{\sin \theta}{\cos^2 \theta} = \frac{2 \cos \theta}{\sin^2 \theta}\]
\[\frac{\sin \theta}{\cos^3 \theta} = \frac{2}{\sin^2 \theta}\]
\[\frac{\sin^3 \theta}{\cos^3 \theta} = 2\]
\[\tan^3 \theta = 2\]
\[\tan \theta = 2^{\frac{1}{3}}\]
\[\theta = \arctan 2^{\frac{1}{3}}\]

\begin{center}
\begin{tabular}{|c|c|c|c|}
\hline
\(\theta\) & \(\left(\arctan 2^{\frac{1}{3}}\right)^-\) & \(\arctan 2^{\frac{1}{3}}\) & \(\left(\arctan 2^{\frac{1}{3}}\right)^+\) \\
\hline
\(L'(\theta)\) & $0^-$ & $0$ & $0^+$ \\
\hline
Shape & $\backslash$ & $-$ & $\slash$ \\
\hline
\end{tabular}
\end{center}

Hence, \(\arctan 2^{\frac{1}{3}}\) is a minimum point. The minimum length would be:
\begin{align*}
L(\arctan 2^{\frac{1}{3}}) &= \frac{1}{\cos \left( \arctan 2^{\frac{1}{3}} \right)} + \frac{2}{\sin \left( \arctan 2^{\frac{1}{3}} \right)} \\
&= 4.161938185 \\
&= \qty{4.16}{\unit{m}}
\end{align*}


\section{Question 7}
\label{sec:orgb073986}

Let the width of the wooden beam be \(w\) and the height of the wooden beam be \(h\). Let the stiffness of the wooden beam be \(s\).
\[s = kwh^3, \quad \text{where } k \text{ is an arbitrary constant}\]

By Pythagoras' Theorem:
\[R^2 = \left( \frac{w}{2} \right)^2 + \left( \frac{h}{2} \right)^2\]
\[\frac{w}{2} = \sqrt{R^2 - \frac{h^2}{4}}\]
\[w = 2\sqrt{R^2 - \frac{h^2}{4}}\]

Hence,
\begin{align*}
s &= 2kh^3 \sqrt{R^2 - \frac{h^2}{4}}, \quad h \in (0, 2R) \\
&= Ch^3 \sqrt{R^2 - \frac{h^2}{4}}, \quad \text{where } C = 2k
\end{align*}

Differentiating \(s\) with respect to \(h\):
\begin{align*}
\frac{ds}{dh} &= 3Ch^2 \sqrt{R^2 - \frac{h^2}{4}} + Ch^3 \frac{1}{\sqrt{R^2 - \frac{h^2}{4}}} \cdot - \frac{2h}{4} \cdot \frac{1}{2} \\
&= 3Ch^2 \sqrt{R^2 - \frac{h^2}{4}} - \frac{Ch^4}{4\sqrt{R^2 - \frac{h^2}{4}}} \\
&= \frac{12Ch^2 \left( R^2 - \frac{h^2}{4} \right) - Ch^4}{4 \sqrt{R^2 - \frac{h^2}{4}}} \\
&= \frac{12Ch^2R^2 - \frac{12Ch^4}{4} - Ch^4}{4 \sqrt{R^2 - \frac{h^2}{4}}} \\
&= \frac{12Ch^2R^2 - 4Ch^4}{4 \sqrt{R^2 - \frac{h^2}{4}}} \\
&= \frac{3Ch^2R^2 - 4Ch^4}{\sqrt{R^2 - \frac{h^2}{4}}} \\
&= \frac{Ch^2(3R^2 - Ch^2)}{\sqrt{R^2 - \frac{h^2}{4}}}
\end{align*}

Finding the stationary points by setting \(\frac{ds}{dh}\) to \(0\):
\[\frac{ds}{dh} = 0\]
\[\frac{Ch^2(3R^2 - Ch^2)}{\sqrt{R^2 - \frac{h^2}{4}}} = 0\]
\[Ch^2(3R^2 - Ch^2) = 0\]
\[3Ch^2R^2 = Ch^4\]
\[h^2 = 3R^2\]
\[h = \sqrt{3}R\]

We see that \(f'(x)\) exists everywhere on \((0, 2R)\) and the only \(x \in (0, 2R)\) where \(f'(x) = 0\) is \(x = \sqrt{3}R\). We also note that:
\[\lim_{x \rightarrow 0} f(x) = \lim_{x \rightarrow 2R} f(x) = 0, \quad \text{when } f \left(\sqrt{3}R \right) \text{ is positive}\]

This makes \(x \sqrt{3}R\) the global maximum on \((0, 2R)\) and the corresponding width is:
\begin{align*}
w &= 2\sqrt{R^2 - \frac{h^2}{4}} \\
&= 2\sqrt{R^2 - \frac{\left(\sqrt{3}R \right)^2}{4}} \\
&= 2\sqrt{R^2 - \frac{3R^2}{4}} \\
&= 2\sqrt{\frac{1}{4} R^2} \\
&= 2 \cdot \frac{1}{2}R^2 \\
&= R
\end{align*}

Hence, the stiffness of the beam is the maximum when the height is \(\sqrt{3}R\) and the width is \(R\).

\newpage

\section{Question 8}
\label{sec:org589f1d1}

Considering the two points \(P_1\) and \(P_2\), which are both distance \(1\) away from the plane interface separating the two mediums. Let \(x\) be the horizontal distance from point \(P_1\) to the vertical line from which the angles \(\theta_1\) and \(\theta_2\) are drawn. Let \(d\) be the distance between the point \(P_1\) and \(P_2\). Furthermore, let \(d_1\) and \(d_2\) be the separation be the distance travelled by light in the 2 mediums.
\\[0pt]

Using Pythagoras' Theorem:
\[d_1^2 = 1 + x^2\]
\[d_2^2 = 1 + (d - x)^2\]

The time taken for the light to travel would be given by:
\begin{align*}
T(x) &= \frac{d_1}{v_1} + \frac{d_2}{v_2} \\
&= \frac{\sqrt{1 + x^2}}{v_1} + \frac{\sqrt{1 + (d - x)^2}}{v_2} \\
&= \frac{\sqrt{1 + x^2}}{v_1} + \frac{\sqrt{1 + d^2 - 2dx + x^2}}{v_2}
\end{align*}

Differentiating with respect to \(x\):
\begin{align*}
T'(x) &= \frac{2x}{v_1\sqrt{1 + x^2}} \cdot \frac{1}{2} + \frac{-2d + 2x}{v_2\sqrt{1 + (d - x)^2}} \cdot \frac{1}{2} \\
&= \frac{x}{v_1 \sqrt{1 + x^2}} + \frac{x - d}{v_2\sqrt{1 + (d - x)^2}}
\end{align*}

\newpage

Finding the stationary points by setting \(T'(x) = 0\):
\[T'(x) = 0\]
\[\frac{x}{v_1 \sqrt{1 + x^2}} + \frac{x - d}{v_2\sqrt{1 + (d - x)^2}} = 0\]
\[\frac{x}{v_1 \sqrt{1 + x^2}} = -\frac{x - d}{v_2\sqrt{1 + (d - x)^2}}\]
\[\frac{x}{v_1 \sqrt{1 + x^2}} = \frac{d - x}{v_2\sqrt{1 + (d - x)^2}}\]
\[xv_2\sqrt{1 + (d - x)^2} = (d - x)v_1\sqrt{1 + x^2}\]
\[v_2\frac{x}{\sqrt{1 + x^2}} = v_1\frac{d - x}{1 + (d - x)^2}\]

Since \(\sin \theta_1 = \frac{x}{\sqrt{1 + x^2}}\) and \(\sin \theta_2 = \frac{d - x}{\sqrt{1 + (d - x)^2}}\):
\begin{align*}
v_2 \sin \theta_1 &= v_1 \sin \theta_2 \\
\frac{\sin \theta_1}{\sin \theta_2} &= \frac{v_1}{v_2} \quad \textbf{(Shown)}
\end{align*}

Differentiating \(T'(x)\) with respect to \(x\):
\begin{align*}
T''(x) &= \frac{1}{v_1} \frac{\sqrt{1 + x^2} - x \frac{1}{\sqrt{1 + x^2}} \cdot \frac{1}{2} \cdot 2x}{1 + x^2}
+ \frac{1}{v_2} \frac{\sqrt{1 + (d - x)^2} - (d - x) \frac{1}{\sqrt{1 + (d - x)^2}} \cdot \frac{1}{2} \cdot (2x - 2d) }{1 + (d - x)^2} \\
&= \frac{1}{v_1} \frac{\sqrt{1 + x^2} - \frac{x^2}{\sqrt{1 + x^2}}}{1 + x^2}
+ \frac{1}{v_2} \frac{\sqrt{1 + (d - x)^2} - \frac{(x - d)(d - x)}{\sqrt{1 + (d - x)^2}}}{1 + (d - x)^2} \\
&= \frac{1}{v_1} \frac{\frac{1 + x^2 - x^2}{\sqrt{1 + x^2}}}{1 + x^2}
+ \frac{1}{v_2} \frac{\frac{1 + (d - x)^2 - (x - d)(d - x)}{\sqrt{1 + (d - x)^2}}}{1 + (d - x)^2} \\
&= \frac{1}{v_1(1 + x^2)^{\frac{3}{2}}}
+ \frac{1 + (d - x)(d - x - (x - d))}{v_2(1 + (d - x)^2)^{\frac{3}{2}}} \\
&= \frac{1}{v_1(1 + x^2)^{\frac{3}{2}}}
+ \frac{1}{v_2(1 + (d - x)^2)^{\frac{3}{2}}} > 0 \text{ for } x \in \mathbb{R} \\
\end{align*}

Since the second derivative of \(T\) is always positive, and we only have 1 critical point for \(T\), that means the critical point is a global minimum. Hence:
\[\frac{\sin \theta_1}{\sin \theta_2} = \frac{v_1}{v_2} \ \ \text{is minimum}\]


\section{Question 9}
\label{sec:org1a5e60b}

\subsection{(a)}
\label{sec:org16d4045}
\begin{align*}
\lim_{x \rightarrow 1} \frac{\ln x}{x - 1} &= \lim_{x \rightarrow 1} \frac{\frac{1}{x}}{1} \\
&= \frac{\frac{1}{1}}{1} \\
&= 1
\end{align*}

\subsection{(b)}
\label{sec:org29d99ed}
\begin{align*}
\lim_{x \rightarrow 0} \frac{\tan x - x}{x^3} &= \lim_{x \rightarrow 0} \frac{\frac{1}{\cos^2 x} - 1}{3x^2} \\
&= \lim_{x \rightarrow 0} \frac{\frac{-2 (- \sin x)}{\cos^3 x}}{6x} \\
&= \lim_{x \rightarrow 0} \frac{\sin x}{3x\cos^3 x} \\
&= \lim_{x \rightarrow 0} \frac{\cos x}{3x \cos^2 x \cdot - \sin x \cdot 3 + 3\cos^3 x} \\
&= \lim_{x \rightarrow 0} \frac{\cos x}{3\cos^3 x - 9x \cos^2 x \sin x} \\
&= \frac{1}{3(1)^3 - 0} \\
&= \frac{1}{3}
\end{align*}

\subsection{(c)}
\label{sec:org3926503}

Using limit laws:
\begin{align*}
\lim_{x \rightarrow \pi} \frac{\sin x}{1 - \cos x} &= \frac{0}{1 - (- 1)} \\
&= 0
\end{align*}

\subsection{(d)}
\label{sec:org9fbacc9}

\begin{align*}
\lim_{x \rightarrow 1} \frac{x^a - 1}{x^b - 1} &= \lim_{x \rightarrow 1} \frac{ax^{a - 1}}{bx^{b - 1}} \\
&= \lim_{x \rightarrow 1} \frac{a}{b} x^{a - 1 - (b - 1)} \\
&= \lim_{x \rightarrow 1} \frac{a}{b} x^{a - b} \\
&= \frac{a}{b}
\end{align*}

\subsection{(e)}
\label{sec:orgde2b67b}

We have
\begin{align*}
\lim_{x \rightarrow \infty} x \ln \left(1 + \frac{a}{x} \right) &= \lim_{x \rightarrow \infty} \left( \frac{\ln (1 + \frac{a}{x})}{\frac{1}{x}} \right) \\
&= \lim_{x \rightarrow \infty} \frac{\frac{1}{1 + \frac{a}{x}} \cdot \frac{-a}{x^2}}{\frac{-1}{x^2}} \\
&= \lim_{x \rightarrow \infty} \frac{a}{1 + \frac{a}{x}} \\
&= a
\end{align*}

Hence:
\begin{align*}
\lim_{x \rightarrow \infty} \left( 1 + \frac{a}{x} \right)^{bx} &= \lim_{x \rightarrow \infty} e^{bx \ln(1 + \frac{a}{x})} \\
&= e^{\lim_{x \rightarrow \infty} b x \ln \left( 1 + \frac{a}{x} \right)} \\
&= e^{ba} \\
&= e^{ab}
\end{align*}

\newpage

\section{Question 10}
\label{sec:org2fed95e}

For \(\varepsilon, p > 0\), we have:
\begin{align*}
\lim_{x \rightarrow \infty} \frac{(\ln x)^p}{x^{\varepsilon}} &= \lim_{x \rightarrow \infty} \left( \frac{\ln x}{x^{\frac{\varepsilon}{p}}} \right)^p \\
&= \lim_{x \rightarrow \infty} \left( \frac{\frac{1}{x}}{\frac{\varepsilon}{p}x^{\frac{\varepsilon}{p} - 1}} \right)^p \\
&= \lim_{x \rightarrow \infty} \left( \frac{1}{\frac{\varepsilon}{p}x^{\frac{\varepsilon}{p}}} \right)^p \\
&= 0^p \\
&= 0 \quad \textbf{(Shown)}
\end{align*}


\section{Question 11}
\label{sec:org5cabd13}
By definition, \(f\) is concave upwards on \(I\) if and only if for all \(a, b \in I\), the line segment joining the points \((a, f(a)), (b, f(b))\) lies above the graph of \(f(x)\).
\\[0pt]

Since the line segment joining the two points is the graph of the function:
\[l(x) = f(a) + \frac{f(b) - f(a)}{b - a}(x - a)\]

We see that \(f\) is concave up on \(I\) if and only if for every \(a < x < b\), we have \(l(x) > f(x)\), or:
\[f(a) + \frac{f(b) - f(a)}{b - a}(x - a) > f(x)\]

Since \(x - a > 0\), we have:
\[f(a) + \frac{f(b) - f(a)}{b - a}(x - a) > f(x)\]
\[\frac{f(b) - f(a)}{b - a}(x - a) > f(x) - f(a)\]
\[\frac{f(b) - f(a)}{b - a} > \frac{f(x) - f(a)}{x - a} \quad \textbf{(Shown)}\]
\end{document}