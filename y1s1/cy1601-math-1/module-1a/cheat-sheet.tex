% Created 2023-08-26 Sat 21:50
% Intended LaTeX compiler: pdflatex
\documentclass[11pt]{article}
\usepackage[utf8]{inputenc}
\usepackage[T1]{fontenc}
\usepackage{graphicx}
\usepackage{longtable}
\usepackage{wrapfig}
\usepackage{rotating}
\usepackage[normalem]{ulem}
\usepackage{amsmath}
\usepackage{amssymb}
\usepackage{capt-of}
\usepackage{hyperref}
\author{Hankertrix}
\date{\today}
\title{Math Module 1A Cheat Sheet}
\hypersetup{
 pdfauthor={Hankertrix},
 pdftitle={Math Module 1A Cheat Sheet},
 pdfkeywords={},
 pdfsubject={},
 pdfcreator={Emacs 29.1 (Org mode 9.6.6)}, 
 pdflang={English}}
\begin{document}

\maketitle
\setcounter{tocdepth}{2}
\tableofcontents

\newpage

\section{Definitions}
\label{sec:org36615fd}

\subsection{Sets}
\label{sec:orgd064ffc}
A \textbf{set} is a collection of objects. These objects are \textbf{points} or \textbf{elements} of the set.

\subsection{Set notation}
\label{sec:org884ce81}
\[A = \{x: \text{some condition}\}\]

\subsection{Subsets}
\label{sec:org9d78380}
If each element of the set \(A\) also belongs to the set \(B\), then \(A\) is a subset of \(B\), which is represented by \(A \subset B\).

\subsection{Equal sets}
\label{sec:orgfdd19a7}
If \(A\) is a subset of \(B\) and \(B\) is also a subset of \(A\), then the sets \(A\) and \(B\) are considered to be \textbf{equal}, which is represented by \(A = B\).

\subsection{Empty set}
\label{sec:org289fe8d}
A set with no elements is called an \textbf{empty set}, and it is denoted by \{\} or \(\emptyset\).
\\[0pt]

Note that an empty \textbf{set} is not the set containing the number 0, as an empty set doesn't contain any elements, while the set containing 0 contains 1 element.

\subsection{Union}
\label{sec:org31802b5}
The union \(A \cup B\) of the sets \(A\) and \(B\) is the set:
\[A \cup B = \{x : x \in A \textbf{ or } x \in B\}\]

\subsection{Intersection}
\label{sec:org42c9483}
The intersection \(A \cap B\) of the sets \(A\) and \(B\) is the set:
\[A \cap B = \{x : x \in A \textbf{ and } x \in B\}\]

\subsection{Subtracting sets}
\label{sec:org94a7f9d}
The set "\(A\) minus \(B\)", written as \(A \setminus B\) is the set:
\[A \setminus B = \{x : x \in A \textbf{ and } x \notin B\}\]

\subsection{Intervals}
\label{sec:org6efdfba}
Generally, an interval is a subset \(I\) of \(\mathbb{R}\) such that:
\[x, y \in I, \quad x < c < y \Rightarrow c \in I\]

\subsection{Interval notation}
\label{sec:org9641510}
The round brackets () means that the number is excluded from the interval while a square bracket means that the number is included in the interval. For example, the interval \((a, b]\) will include the number \(a\), but will include the number \(b\).

\subsection{Statements}
\label{sec:orgfe3ae19}
Statements in maths are \textbf{either} true or false. They cannot be both true and false, or neither true nor false.

\subsection{Negation of a statement}
\label{sec:orgec4c987}
The \textbf{negation} \(not \ P\) of a statement \(P\), is a statement that is \textbf{true} when \(P\) is \textbf{false}.

\subsection{"For all" statements}
\label{sec:orgde7a695}
"For all" statements are usually represented with the symbol \(\forall\). Generally, the negation of a "for all" statement is a "there exists" statement.

\subsection{"There exists" statements}
\label{sec:org4371ef8}
"There exists" statements are usually represented with the symbol \(\exists\). Generally, the negation of a "there exists" statement is a "for all" statement.

\subsection{Open sets}
\label{sec:orgecb2e67}
A set is \(A \subset \mathbb{R}\) is \textbf{open}, if \textbf{for every \(x \in A\), there exists \(\delta > 0\) such that \((x - \delta, x + \delta) \subset A\)}. If not, \(A\) is not open.

\subsection{Not open sets}
\label{sec:orgfc243aa}
A set is \textbf{not open} means that *there exists \(x \in A\) such that for every \(\delta > 0\), \((x - \delta, x + \delta) \not\subset A\).

\subsection{Closed sets}
\label{sec:org89df035}
A set \(A \subset \mathbb{R}\) is \textbf{closed} if its "complement" \(\mathbb{R} \setminus A\) is open.

\subsection{Clopen sets}
\label{sec:orga26804f}
Open and closed sets are not mutually exclusive, which means that a set can be both open and closed at the same time. Sets that are both open and closed at the same time, like the empty set \(\emptyset\) and the set of real numbers \(\mathbb{R}\), are called \textbf{clopen} sets.

\subsection{Logical AND}
\label{sec:org088fb43}
Given two statements \(P\) and \(Q\), the statement \(P\) AND \(Q\) is true when \textbf{both} \(P\) and \(Q\) are \textbf{true}, and false otherwise.

\subsection{Logical OR}
\label{sec:org5102011}
Given two statements \(P\) and \(Q\), the statement \(P\) OR \(Q\) is true when \textbf{both} \(P\) and \(Q\) are \textbf{false}, and true otherwise.

\subsection{Implications}
\label{sec:org56007ac}
Implications are statements that have the form "if something, then something". Implications are usually represented with the symbol \(\Rightarrow\).
\\[0pt]

So the statement "if P, then Q" can be expressed as:
\[P \Rightarrow Q \text{ which also means } P \text{ implies } Q\]

An implication has a "direction", which means that if \(P\) implies \(Q\), that does not necessarily mean that \(Q\) also implies \(P\).

\subsubsection{Another way of to think of implications}
\label{sec:org79fa0de}
The implication \(P \Rightarrow Q\) is another way of saying \(not \ (P \text{ AND } not \ Q)\).

\subsection{Equivalences}
\label{sec:org44d4c37}
Two statements \(P\) and \(Q\) are equivalent when \(P\) if and only if \(Q\). This can be represented as:

\[(P \Rightarrow Q) \text{ and } (Q \Rightarrow P)\]
\[P \Leftrightarrow Q\]

\subsection{Functions}
\label{sec:org8390247}
Consider two sets \(A\) and \(B\). A \textbf{function} \(f : A \rightarrow B\) is a rule that assigns to each element \(x \in A\) exactly one element \(f(x) \in B\) called the value of the function \(f\) at the point \(x\).
\\[0pt]

Put simply, a function takes a set of inputs, \(A\) and returns a set of outputs \(B\). One input can only have one output.
\\[0pt]

We also say that \(f : A \rightarrow B\) is a function \textbf{from} A \textbf{to} B.

\subsection{Domain of a function}
\label{sec:org0f3cb93}
For a function \(f : A \rightarrow B\), the set \(A\) is called the \textbf{domain} of \(f\). Note that for a function \(f\) with domain \(A\), \(f(x)\) is only defined for \(x \in A\). The domain of a function is also part of the definition of a function, so the two functions below are different functions.

\[f : \mathbb{R} \rightarrow \mathbb{R} \ f(x) = x^2\]
\[g : [0, \infty) \rightarrow \mathbb{R} \ g(x) = x^2\]

\subsection{Codomain of a function}
\label{sec:org24a29c6}
For a function \(f : A \rightarrow B\), the set \(B\) is called the \textbf{codomain} of \(f\).

\subsection{Sequences}
\label{sec:orgffbb958}
A function \(f : A \rightarrow \mathbb{R}\) where \(A\) is a subset of \(\mathbb{N}\) is called a \textbf{sequence}. Usually, sequences are notated using \(a_n\) and \((a_n), (a_n)_{n=1}^{\infty}, \ (a_1, a_2, a_3, ...), \ etc.\) instead of \(f(n)\) and \(f\) respectively.

\subsection{Different ways of describing functions}
\label{sec:org06f5813}
\begin{enumerate}
\item \textbf{Explicit} formulae like \(f(x) = \sin (1 + x^3)\) and \(a_n = 2^n\)
\item \textbf{Implicit} formulae like \(\sin g(t) = t, \quad -\frac{\pi}{2} \le g(t) \le \frac{\pi}{2}\)
\item \textbf{Recurrent} formulae for a \textbf{sequence} like \(a_1 = 2, a_{n + 1} = 2a_n\)
\end{enumerate}

\subsection{Natural domain of a function}
\label{sec:orgf92569f}
When nothing is said about the domain of a function \(f\), the largest set \(A\) for which \(f(x)\) "makes sense" \(x \in A\) is called the \textbf{natural domain} of the function \(f\).

\subsection{Image of a function}
\label{sec:orgacddb99}
Consider a function \(f : A \rightarrow B\). For \(K \subset A\), the set \(f(K) = \{f(x) : x \in K\} \subset B\) is called the \textbf{image} of the set \(K\).

\subsection{Range of a function}
\label{sec:orgd5c168e}
Consider a function \(f : A \rightarrow B\). The image \(f(A)\) of the whole domain \(A\) is called the \textbf{range} of the function \(f\). This means that the range of a function is an image of the function for the \textbf{entire domain} of the function.

\subsection{Increasing functions}
\label{sec:orgc3ed40e}
A function \(f : A \rightarrow \mathbb{R}\) where \(A \subset \mathbb{R}\), is said to be *increasing if:
\[x_1, x_2 \in A, \  x_1 < x_2 \Rightarrow f(x_1) \le f(x_2)\]

\subsection{Decreasing functions}
\label{sec:org6ff125f}
A function \(f : A \rightarrow \mathbb{R}\) where \(A \subset \mathbb{R}\), is said to be \textbf{decreasing} if:
\[x_1, x_2 \in A, \  x_1 < x_2 \Rightarrow f(x_1) \ge f(x_2)\]

\subsection{Monotonic functions}
\label{sec:org5e0b36f}
Monotonic functions are functions that are either increasing or decreasing.

\subsection{Strictly increasing functions}
\label{sec:org20ee6e6}
A function \(f : A \rightarrow \mathbb{R}\) where \(A \subset \mathbb{R}\), is said to be \textbf{strictly increasing} if:
\[x_1, x_2 \in A, \  x_1 < x_2 \Rightarrow f(x_1) < f(x_2)\]

\subsection{Strictly decreasing functions}
\label{sec:org04a6e21}
A function \(f : A \rightarrow \mathbb{R}\) where \(A \subset \mathbb{R}\), is said to be \textbf{strictly decreasing} if:
\[x_1, x_2 \in A, \  x_1 < x_2 \Rightarrow f(x_1) > f(x_2)\]

\subsection{Strictly monotonic functions}
\label{sec:org4d8f4c8}
Strictly monotonic functions are function that are either strictly increasing or strictly decreasing.

\subsection{Bounded functions}
\label{sec:org119d3f2}
A function \(f : A \rightarrow \mathbb{R}\) is \textbf{bounded} if there exists an \(M > 0\) such that:
\[|f(x)| \le M, \ \text{ for all } x \in A\]

Note that:
\[|f(x)| \le M \Leftrightarrow -M \le f(x) \le M\]

A function that is not bounded is said to be \textbf{unbounded}.

\subsection{Locally bounded functions}
\label{sec:org6fc4282}
A function \(f : A \rightarrow \mathbb{R}\) is \textbf{locally bounded at point} \(a \in A\) if there exists some \(\delta > 0\) such that \(f\) is bounded on \(A \cap (a - \delta, a + \delta)\).
\\[0pt]

A function that is \textbf{locally bounded} means that \(f\) is locally bounded at every \(a \in A\).

\subsection{Unbounded functions}
\label{sec:orgaa96f59}
A function \(f : A \rightarrow \mathbb{R}\) is unbounded if for all \(M > 0\), there exists \(x \in A\) such that \(|f(x)| > M\).

\subsection{Odd functions}
\label{sec:orga77cba0}
A function \(f : A \rightarrow \mathbb{R}\) is said to be \textbf{odd} if:
\[x \in A \quad \Rightarrow \quad -x \in A \text{ and } f(-x) = -f(x)\]

\subsection{Even functions}
\label{sec:orgad85394}
A function \(f : A \rightarrow \mathbb{R}\) is said to be \textbf{even} if:
\[x \in A \quad \Rightarrow \quad -x \in A \text{ and } f(-x) = f(x)\]

\subsection{Periodic functions}
\label{sec:org42046c5}
A function \(f: \mathbb{R} \rightarrow \mathbb{R}\) is said to be \textbf{periodic} if there exists some \(T > 0\) such that:
\[f(x + T) = f(x), \quad \text{ for all } x \in \mathbb{R}\]

The number \(T\) is called a \textbf{period} for \(f\).


\section{Standard sets}
\label{sec:org2b9f610}
\[\mathbb{N} = \{1, 2, 3, 4, ...\} \text{ is the set of natural numbers}\]
\[\mathbb{Z} = \{..., -2, -1, 0, -1, -2, ...\} \text{ is the set of integers}\]
\[\mathbb{Q} = \{\frac{m}{n} : m \in \mathbb{Z}, n \in \mathbb{N}\} \text{ is the set of rational numbers}\]
\[\mathbb{R} \text{ is the set of real numbers}\]

\newpage

\section{Explanation of an open set}
\label{sec:org8864a74}
A set \(A \subset \mathbb{R}\) is open if for every \(x \in A\), there exists a \(\delta > 0\) such that \((x - \delta, x + \delta) \subset A\). The set \(A\) is not open when there exists \(x \in A\) such that for every \(\delta > 0\), \((x - \delta, x + \delta) \notin A\).
\\[0pt]

To make this definition easier to understand, let's assume \(\delta\) to be a very small number. Let's look at the set of \((3,5)\). If we pick a \(x\) value that is very close to the boundary, like 4.9999 (\(x \neq 5\) as the set doesn't include 5), there's still a value of \(\delta\) that is greater than 0 (\(\delta > 0\)) that can be added to 4.9999 that will not cause \((x + \delta)\) to exceed the bounding value. In this case, \(\delta\) would be 0.000001. Similarly, for the lower bound, we can pick a \(x\) value that is very close to the boundary, such as 3.0001 (\(x \neq 3\) as the set doesn't include 3), and there will still be a non-zero \(\delta\) that is greater than 0 (\(\delta > 0\)) that can be subtracted from \(x\) that will not cause \((x - \delta)\) to be lower than the lower bound. In this case, \(\delta\) would be 0.00001. Hence, \((3,5)\) would be an open set.
\\[0pt]

Now, let's look at the set of \([3,5]\). If we pick a \(x\) value that is at the boundary (remember that the set includes the boundaries), like 5, there's no value of \(\delta\) that is greater than 0 (\(\delta > 0\)) that can be added to 5 that will not cause \((x + \delta)\) to exceed the bounding value. Similarly, for the lower bound, if we pick a \(x\) value that is at the boundary (the set includes the boundaries), such as 3, and there's no value of \(\delta\) that is greater than 0 (\(\delta > 0\)) that can be subtracted from \(x\) that will not cause \((x - \delta)\) to be lower than the lower bound. Hence, \((3,5)\) would not be an open set.
\end{document}