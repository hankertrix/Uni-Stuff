% Created 2023-08-29 Tue 12:37
% Intended LaTeX compiler: pdflatex
\documentclass[11pt]{article}
\usepackage[utf8]{inputenc}
\usepackage[T1]{fontenc}
\usepackage{graphicx}
\usepackage{longtable}
\usepackage{wrapfig}
\usepackage{rotating}
\usepackage[normalem]{ulem}
\usepackage{amsmath}
\usepackage{amssymb}
\usepackage{capt-of}
\usepackage{hyperref}
\author{Hankertrix}
\date{\today}
\title{Math Module 1B Cheat Sheet}
\hypersetup{
 pdfauthor={Hankertrix},
 pdftitle={Math Module 1B Cheat Sheet},
 pdfkeywords={},
 pdfsubject={},
 pdfcreator={Emacs 29.1 (Org mode 9.6.6)}, 
 pdflang={English}}
\begin{document}

\maketitle
\setcounter{tocdepth}{2}
\tableofcontents

\newpage

\section{Definitions}
\label{sec:org6abed0f}

\subsection{Distance between points}
\label{sec:org2f1acc5}
The distance between two points \(x, y \in \mathbb{R}\) is \(|x - y|\).

\subsection{Limit points}
\label{sec:org03814fe}
For \(A \subset \mathbb{R}\), a point \(a \in \mathbb{R}\) is a \textbf{limit point} of \(A\) if for every \(\delta >0\), there exists a point \(x \n A\) such that \(0 < |x - a| < \delta\).

\subsection{Limit}
\label{sec:orge267753}
For a function \(f : A \rightarrow \mathbb{R}, \, A \subset \mathbb{R}\) with \(a\) as a limit point of \(A\), \(f\) approaches a \textbf{limit} \(L\) if for every \(\varepsilon > 0\), there exists a \(\delta > 0\) such that:

\[\lim_{x \rightarrow a} f(x) = L\]
\[\Updownarrow\]

For every \(\varepsilon > 0\), there exists a \(\delta > 0\) such that:
\[0 < |x - a| < \delta, \quad x \in A \ \ \Rightarrow \ \ |f(x) - L| < \varepsilon.\]

\subsection{Limits at infinity}
\label{sec:orga092976}
Suppose \(f\) is defined on some interval \((a, \infty)\). We say that \(f(x)\) has a limit \(L\) as \(x\) approaches positive infinity, and write \(\lim_{X \rightarrow + \infty} f(x) = L\), if for every \(\varepsilon > 0\), there exists a number \(R\) such that:
\[x > R \quad \Rightarrow \quad |f(x) - L| < \varepsilon\]

Likewise, for \(f\) defined on some interval \((-\infty, b)\), we say that \(f(x)\) has a limit \(L\) as \(x\) approaches negative infinity, and write \(\lim_{x \rightarrow - \infty} f(x) = L\), if for every \(\varepsilon > 0\), there exists a number \(R\) such that:
\[x < R \quad \Rightarrow \quad |f(x) - L| < \varepsilon\]

Limits at infinity follow the same limit laws as normal limits, so we can use limit laws to conclude that for any \textbf{positive} integer \(n\), we also have:
\[\lim_{x \rightarrow \pm \infty} \frac{1}{x^n} = 0\]

When evaluating a limit at infinity, a common technique is to factor out the highest possible power.

\subsection{Limit of a sequence}
\label{sec:org45213da}
We say that a sequence \((a_n)\) has the limit \(L\) and write \(\lim_{n \rightarrow \infty} a_n = L\), if for every \(\varepsilon > 0\), there exists a number \(N\) such that:
\[n > N \quad \Rightarrow \quad |a_n - L| < \varepsilon\]

The limits of sequences are evaluated with similar methods to other forms of limits.


\section{Limit laws}
\label{sec:org38b85c6}
Consider \(f : A_1 \rightarrow \mathbb{R}, g : A_2 \rightarrow \mathbb{R}\). Suppose \(a\) is a limit point of \(A_1 \cap A_2\), and \(\lim_{x \rightarrow a} \, f(x) = l, \lim_{x \rightarrow a} \, g(x) = m\), then:

\begin{align*}
\textbf{1. } \lim_{x \rightarrow a}(Af(x) + Bg(x)) &= Al + Bm \\
&= A \cdot \lim_{x \rightarrow a} f(x) + B \cdot \lim_{x \rightarrow a} g(x)
\end{align*}

\begin{align*}
\textbf{2. } \lim_{x \rightarrow a}(f(x) g(x)) &= lm \\
&= \lim_{x \rightarrow a} f(x) \cdot \lim_{x \rightarrow a} g(x)
\end{align*}

\begin{align*}
\textbf{3. } \lim_{x \rightarrow a} \frac{f(x)}{g(x)} &= \frac{l}{m},
\text{ provided } m \neq 0 \\
&= \frac{\lim_{x \rightarrow a} f(x)}{\lim_{x \rightarrow a} g(x)}
\end{align*}

\begin{align*}
\textbf{4. } \lim_{x \rightarrow a} \sqrt[n]{f(x)} &= \sqrt[n]{l},
\text{ provided } n \in \mathbb{N} \text{ and } l \ge 0 \text{ if } n \text{ is even} \\
&= \sqrt[n]{\lim_{x \rightarrow a} f(x)}
\end{align*}

\textbf{5.} L'H\(\text{\^o}\)pital's rule:
\[\lim_{x \rightarrow a} \frac{f(x)}{g(x)} = \lim_{x \rightarrow a} \frac{f'(x)}{g'(x)}, \text{ when } \lim_{x \rightarrow a} \frac{f(x)}{g(x)} = \frac{0}{0} \text{ and } g'(x) \neq 0\]

\section{Squeeze Theorem}
\label{sec:org7faca05}
Suppose \(f(x) \leq g(x) \leq h(x)\), for \(x \in I \setminus \{a\}\), where \(I\) is some open interval containing the point \(a\). Then:

\[\lim_{x \rightarrow a} f(x) = \lim_{x \rightarrow a} h(x) = L \quad \Rightarrow \quad \lim_{x \rightarrow a} g(x) = L\]

\subsection{Extremely useful result}
\label{sec:orgb77a125}
For \(f : A \rightarrow \mathbb{R}\), we have:
\[\lim_{x \rightarrow a} f(x) = L \quad \Leftrightarrow \quad \lim_{x \rightarrow a} |f(x) - L| = 0\]

\subsection{A useful lemma}
\label{sec:orgb3244bf}
For \(0 < x < \frac{\pi}{2}\), we have:
\[x \cos^2 x < \sin x < x\]

If \(f\) and \(g\) are \textbf{even} functions such that \(f(x) < g(x)\), for \(x \in (0, a)\), then we also have:
\[f(x) < g(x), \text{ for } x \in (-a, 0)\]

\section{Useful limits}
\label{sec:orge66aac9}
\[\lim_{x \rightarrow 0} \sin x = 0\]
\[\lim_{x \rightarrow 0} \cos x = 1\]
\[\lim_{x \rightarrow 0} \frac{\sin x}{x} = 1\]
\end{document}