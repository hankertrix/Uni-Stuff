% Created 2025-04-29 Tue 19:47
% Intended LaTeX compiler: pdflatex
\documentclass[11pt]{article}
\usepackage[utf8]{inputenc}
\usepackage[T1]{fontenc}
\usepackage{graphicx}
\usepackage{longtable}
\usepackage{wrapfig}
\usepackage{rotating}
\usepackage[normalem]{ulem}
\usepackage{amsmath}
\usepackage{amssymb}
\usepackage{capt-of}
\usepackage{hyperref}
\usepackage{minted}
\author{Hankertrix}
\date{\today}
\title{Math Module 4B Notes}
\hypersetup{
 pdfauthor={Hankertrix},
 pdftitle={Math Module 4B Notes},
 pdfkeywords={},
 pdfsubject={},
 pdfcreator={Emacs 30.1 (Org mode 9.7.11)}, 
 pdflang={English}}
\begin{document}

\maketitle
\setcounter{tocdepth}{2}
\tableofcontents \clearpage\section{Definitions}
\label{sec:orgbf52b5f}

\subsection{Differential equation (d.e or DE)}
\label{sec:org0aa2e59}
A differential equation is an equation involving one or more derivatives of an unknown function. The order of the highest derivative occurring in a differential equation is called the \textbf{order} of the differential equation.
\subsubsection{Examples}
\label{sec:orgf18f6a5}
\[\frac{dy}{dx} + x^2 = y\]
\[\frac{d^2y}{dx^2} = -k^2 y\]
\[\frac{d^3y}{dx^3} + \left( \frac{d^2y}{dx^2} \right)^5 + \cos x = 0\]
\[\sin \left( \frac{dy}{dx} \right) + \arctan y = 1\]

The equations above are all differential equations in the unknown function \(y(x)\).
\subsection{Solution to a differential equation}
\label{sec:orge0fa9d4}
Consider an \(n\)th order differential equation in the unknown \(y\). A function \(y(x)\) which is (at least) \(n\) times differentiable on an interval \(I\) is called a \textbf{solution} to the differential equation on \(I\), if the substitution \(y = y(x), y' = y'(x), \ldots, y^{(n)} = y^{(n)}(x)\) reduces the differential equation to an identity valid for all \(x \in I\).


A solution to a differential equation is sometimes also called a \textbf{particular solution}.


The \textbf{general} solution to a differential equation is the collection of all (particular) solutions.

\newpage
\subsection{Initial-value problems (i.v.p)}
\label{sec:org6c8f988}
An \(n\)th-order differential equation together with \(n\) \textbf{initial conditions} of the form:
\[y(x_0) = y_0\]
\[y'(x_0) = y_0\]
\[\vdots\]
\[y^{(n - 1)}(x_0) = y_{n - 1}\]

Where \(y_0, y_1, \ldots, y_{n-1}\) are constants, is called an \textbf{initial-value problem}. A solution to an initial value problem is a function that satisfies both the stated differential equations and all the initial conditions.
\subsection{Separable differential equations}
\label{sec:org1ca15d9}
A first-order differential equation is called \textbf{separable} if it can be written in the form:
\[p(y) y' = q(x)\]

A separable differential equation is just a differential equation that we can separate \(x\) and \(y\).


If \(p\) and \(q\) are continuous, we can solve \(p(y) y' = q(x)\) by integrating both sides with respect to \(x\):
\begin{align*}
\int p(y) y' \, dx &= \int q(x) \, dx \\
\int p(y) \, dy &= \int q(x) \, dx
\end{align*}

\newpage
\subsection{Linear differential equation}
\label{sec:orgc6dcd66}
A differential equation that can be written in the form:
\[a_0 (x) y^{(n)} + a_1 (x) y^{(n - 1)} + \cdots + a_n (x) y = F(x)\]

Where \(a_0, a_1, \ldots, a_n\) and \(F\) are functions of \(x\) only, is called a \textbf{linear} differential equation of order \(n\).


Basically, the degree of all terms in \(y\) is \textbf{at most one}.
\subsubsection{Examples}
\label{sec:orged4d099}
\[I \cdot y'' + x^2 y + (\sin x) y = e^x \tag{1}\]
\[xy''' + 4x^2 y' - \frac{2}{1+ x^2} y = 0 \tag{2}\]

The equations above (\((1) \text{ and } (2)\)) are \textbf{linear} differential equations.

\[y'' + x \sin (y') - xy = x^2 \tag{3}\]
\[y'' + x^2y' + y^2 = 0 \tag{4}\]

The equations above (\((3) \text{ and } (4)\)) are \textbf{nonlinear} differential equations.
\subsection{Linear initial value problems}
\label{sec:org8654998}
Let \(a_1, a_2, \ldots, a_n, F\) be functions that are continuous on an interval \(I\). Then, for any \(x_0 \in I\), the initial-value problem below has a unique solution on \(I\).
\[y^{(n)} + a_1 (x) y^{(n - 1)} + \cdots + a_{n - 1}(x) y' + a_n (x) y = F(x)\]
\[y(x_0) = y_0, y'(x_0) = y_1, \ldots, y^{(n - 1)} x_0 = y_{n - 1}\]

\newpage
\subsection{First-order linear differential equations}
\label{sec:org3005e13}
A differential equation that can be written in the form:
\[a(x) \frac{dy}{dx} + b(x)y = r(x)\]

Where \(a(x), b(x), \text{ and } r(x)\) are functions defined on an interval \((a, b)\), is called a \textbf{first-order linear differential equation}.


If \(a(x) \ne 0\) on \((a, b)\), we have the \textbf{standard form}:
\[\frac{dy}{dx} + p(x) y = q(x)\]

Where \(p(x) = \frac{b(x)}{a(x)} \text{ and } q(x) = \frac{r(x)}{a(x)}\).


Let \(P(x)\) be an antiderivative to \(p(x)\) and multiply with \(e^{P(x)}\) to get:
\[e^{P(x)} \frac{dy}{dx} + p(x) e^{P(x)} y = q(x) e^{P(x)}\]
\[\frac{d}{dx} \left( e^{P(x)} y \right) = q(x) e^{P(x)}\]

We can solve the problem by integration.
\subsection{Homogeneous differential equation}
\label{sec:org2690fb5}
Homogeneous just means that the differential equation is equal to 0.

\newpage
\subsection{Second-order linear differential equation}
\label{sec:org12f37ed}
A \textbf{second-order linear} differential equation, has the form:
\[a_0(x)y'' + a_1(x)y' + a_2(x)y = F(x)\]

Here, \(a_0, a_1, a_2 \text{ and } F\) are functions defined on an interval \(I\). If \(F(x) = 0\) for all \(x \in I\), we say that the equation is \textbf{homogeneous}.


If \(a_0 (x) \ne 0\) on \(I\), dividing gives the \textbf{standard form}:
\[y'' + p(x)y' + q(x)y = f(x)\]
\subsubsection{Theorem}
\label{sec:orga06f90a}
For a \textbf{homogeneous} linear differential equation:
\[a(x)y'' + b(x)y' + c(x)y = 0\]

If \(y_1(x)\) and \(y_2(x)\) are two solutions on the interval \(I\), then any \textbf{linear combination} below is also a solution on \(I\):
\[y(x) = C_1 y_1 (x) + C_2 y_2 (x)\]

Where \(C_1, C_2\) are constants.


The linearity principle holds \textbf{only} for differential equations that are \textbf{both homogeneous} and \textbf{linear}. The result is stated above for a second-order linear homogeneous equation, but the analogous result holds for \(n\)th order linear homogeneous equations.

\newpage
\subsection{Linearly dependent functions}
\label{sec:orge1f1de2}
Two functions defined on the interval \(I\) are said to be \textbf{linearly dependent} on \(I\), if one function is a \textbf{scalar (constant) multiple} of another function.
\subsubsection{Example}
\label{sec:orgeaa423c}
\[y_1(x) = \cos 2x\]
\[y_2(x) = 3(1 - 2 \sin^2 x)\]

\begin{align*}
y_2(x) &= 3 (1 - 2 \sin^2 x) \\
&= 3 \left( 1 - 2 \cdot \frac{1 - \cos 2x}{2} \right) \\
&= 3 (1 - (1 - \cos 2x)) \\
&= 3 \cos 2x \\
&= 3y_1(x)
\end{align*}

Since \(y_2(x) = 3y_1(x), y_1(x) \text{ and } y_2(x)\) are \textbf{linearly dependent}.

\newpage
\subsection{Linearly independent functions}
\label{sec:orgfd91929}
Two functions defined on the interval \(I\) are said to be \textbf{linearly independent} if one function is \textbf{not} a scalar (constant) multiple of another function.
\subsubsection{Example}
\label{sec:orgabc616e}
\[y_1(x) = e^x\]
\[y_2(x) = xe^x\]

Since neither \(e^x\) nor \(xe^x\) is a constant multiple of the other, the two functions are \textbf{linearly independent}.
\subsubsection{Theorem}
\label{sec:org816689a}
Let \(I\) be an interval and consider the equations:
\[y'' + p(x)y' + q(x)y = f(x) \tag{3}\]
\[y'' + p(x)y' + q(x)y = 0 \tag{4}\]

Let \(y_1(x), y_2(x)\) be \textbf{linearly independent} solutions of \((4)\) and \(y_p(x)\) a solution of \((3)\) on \(I\). Then:
\begin{itemize}
\item The general solution of \((4)\) on \(I\) is:
\[y(x) = C_1 y_1(x) + C_2 y_2(x), \quad C_1, C_2 \in \mathbb{R}\]
\item The general solution to \((3)\) on \(I\) is:
\[y(x) = C_1 y_1(x) + C_2 y_2(x) + y_p(x), \quad C_1, C_2 \in \mathbb{R}\]
\end{itemize}

So, the \textbf{general solution} to the non-homogeneous differential equation \(y'' + p(x)y' + q(x)y = f(x)\) is of the form:
\[y(x) = y_c(x) + y_p(x)\]

Where \(y_c(x) = C_1 y_1(x) + C_2 y_2(x)\) is the general solution to the associated homogeneous equation:
\[y'' + p(x)y' + q(x)y = 0\]

And \(y_p\) is a particular solution to:
\[y'' + p(x)y' + q(x)y = f(x)\]
\subsection{Characteristic equation}
\label{sec:org8ece60b}
Consider a homogeneous linear differential equation of order 2 with \textbf{constant coefficients}.
\[ay'' + by' + cy = 0\]

Where \(a, b, c \in \mathbb{R}, a \ne 0\).


Since, in order to satisfy the equation, we want constant multiples of \(y\) and its derivatives to cancel, we should look for solutions of this differential equation of the form \(y(x) = e^{rx}\) (since derivatives of \(y\) are constant multiples of \(y\)). Trying this:


With \(y(x) = e^{rx}\), we get:
\[y'(x) = re^{rx}, \ y''(x) = r^2 e^{rx}\]

Substituting the above equation into the differential equation \(ay'' + by' + cy = 0\), we get:
\[ar^2 e^{rx} + bre^{rx} + ce^{rx} = 0\]
\[e^{rx} (ar^2 + br + c) = 0\]

Since \(e^{rx} > 0\), the equation is satisfied only if:
\[ar^2 + br + c = 0\]

The quadratic equation above is called the \textbf{characteristic equation} for the differential equation above.

\newpage
\section{Solving first-order differential equations (FODEs)}
\label{sec:orgea8da41}
In order to solve first-order differential equations of the form:
\[\frac{dy}{dx} + p(x)y = q(x) \tag{1}\]

We will use a method called the integrating factor. Introducing an integrating factor called \(\mu\) to the equation above:
\[\mu \frac{dy}{dx} + \mu p(x)y = \mu q(x)\]

Considering the product rule to simplify the above equation:
\[\frac{d}{dx} (uv) = u \frac{du}{dx} + v \frac{du}{dx}\]

Comparing this with the left-hand side of the original first-order differential equation, we have:
\[u = \mu \tag{2}\]
\[\frac{du}{dx} = \frac{dy}{dx} \tag{3}\]
\[v = y \tag{4}\]
\[\frac{du}{dx} = \mu p(x) \tag{5}\]

Equations \((2)\) and \((5)\) yield:
\[\frac{d \mu}{dx} = \frac{du}{dx} = \mu p (x)\]

Solving this with the separation of variables:
\[\frac{d \mu}{dx} = \mu p(x)\]
\[\frac{1}{\mu} \frac{d \mu}{dx} = p(x)\]
\[\int \frac{1}{\mu} \, d \mu = \int p(x) \, dx\]
\[\ln |\mu| = \int p(x) \, dx\]
\[\mu = e^{\int p(x) \, dx} \tag{6}\]

Substituting \((6)\) into \((1)\):
\[e^{\int p(x) \, dx} \frac{dy}{dx} + e^{\int p(x) \, dx} y = e^{\int p(x) \, dx} q(x)\]

Using equations \((2), (3), (4), (5) \text{ and } (6)\) with the product rule:
\[\frac{d}{dx} \left( e^{\int p(x) \, dx} y \right) = e^{\int p(x) \, dx} q(x)\]

Integrating both sides with respect to \(x\), we get:
\[e^{\int p(x) \, dx} y = \int q(x) e^{\int p(x) \, dx} \, dx\]

In summary, reduce the given first-order differential equation into the form \(\frac{dy}{dx} + p(x) y = q(x)\), then find the integrating factor with \(\mu = e^{\int p(x) \, dx}\) and multiply every term by it. Apply the product rule to obtain \(\frac{d}{dx} \left( e^{\int p(x) \, dx} y \right) = e^{\int p(x) \, dx} q(x)\). Then integrate both sides with respect to \(x\) and solve for \(y\).
\section{Solving second-order differential equations (SODEs)}
\label{sec:org4767fdb}

\subsection{Solving linear homogeneous second-order differential equations}
\label{sec:org473847a}
To solve linear (degree of all terms is at most 1) homogeneous (the equation is equal to 0) second-order differential equations of the form:
\[a \frac{d^2y}{dx^2} + b \frac{dy}{dx} + cy = 0\]

We are finding the \textbf{general solution} of \(y\). First, we need to identify the \textbf{characteristic or auxiliary equation} of the second-order differential equation. It is given by:
\[am^2 + bm + c = 0\]

Using the quadratic formula,
\[m = \frac{-b \pm \sqrt{b^2 - 4ac}}{2a}\]

We now have three cases for the different types of roots.
\subsubsection{Case 1: Roots are real and distinct (\(b^2 - 4ac > 0\))}
\label{sec:orge4a3f2a}
The general solution is:
\[y = C_1 e^{m_1 x} + C_2 e^{m_2 x}\]

Where \(C_1\) and \(C_2\) are constants to be found.
\subsubsection{Case 2: Roots are real and equal (\(b^2 - 4ac = 0\))}
\label{sec:org6d5d8da}
The general solution is:
\[y = (C_1 + C_2 x) e^{mx}\]
\subsubsection{Case 3: Roots are complex (\(b^2 - 4ac < 0\))}
\label{sec:orga6890e9}
The general solution is:
\[y = e^{\alpha x} (C_1 \cos (\beta x) + C_2 \sin (\beta x)), \quad m = \alpha + \beta i\]
\subsection{Solving linear non-homogeneous second-order differential equations}
\label{sec:org9dad825}
To solve linear non-homogeneous second-order differential equations of the form:
\[a \frac{d^2y}{dx^2} + b \frac{dy}{dx} + cy = f(x)\]

We must first find the \textbf{complimentary function}, which is the function \(y = q(x)\). When this function is substituted into the second-order differential equation, the right-hand side is 0 (similar to the \textbf{general solution} of a linear homogeneous second-order differential equation). After which, we must find the \textbf{particular solution}, which is the function \(y = p(x)\). When this function is substituted into the second-order differential equation, it gives us \(f(x)\). Finally, the general solution to the linear non-homogeneous second-order differential equation is given by:
\[y = \textbf{Complimentary function} + \textbf{Particular solution}\]

To find the particular solution, we must consider 3 cases.
\subsubsection{Case 1: \(f(x)\) is a polynomial of degree \(n, f(x) = a_0 + a_1 x + \ldots + a_n x^n\)}
\label{sec:org0ba081f}
The particular solution is a polynomial with degree equal to the degree of \(f(x)\).
\[p(x) = b_0 + b_1 x + b_2 x^2 + \ldots + b_n x^n\]
\subsubsection{Case 2: \(f(x) = (c_0 + c_1 x + c_2 x^2 + \ldots + c_n x^n) e^{kx}, \  c_n \in \mathbb{R}\)}
\label{sec:orgbdd16d6}

\begin{enumerate}
\item The complimentary function does not have \(e^{kx}\)
\label{sec:org5cac47a}

The particular solution is:
\[p(x) = (c_0 + c_1 x + c_2 x^2 + \ldots + c_n x^n) e^{kx}\]
\item The complimentary function has \(e^{kx}\) but not \(xe^{kx}\)
\label{sec:orgc22a7ff}

The particular solution is:
\[p(x) = x(c_0 + c_1 x + c_2 x^2 + \ldots + c_n x^n) e^{kx}\]
\item The complimentary function has \(e^{kx}\) and \(xe^{kx}\)
\label{sec:orgde4a965}

The particular solution is:
\[p(x) = x^2(c_0 + c_1 x + c_2 x^2 + \ldots + c_n x^n) e^{kx}\]
\item The complimentary function is \(q(x) = e^{\alpha x} (C_1 \cos (\beta x) + C_2 \sin (\beta x))\)
\label{sec:org75a3c50}

The particular solution is:
\[p(x) = pe^{kx}\]
\end{enumerate}
\subsubsection{Case 3: \(f(x) = k \cos (ax), k \sin (ax) \text{ or } k \cos (ax) + r \sin (ax)\)}
\label{sec:orgf563f7f}

\begin{enumerate}
\item The complimentary function does not have \(A \cos (ax) + B \sin (ax)\)
\label{sec:org8c68c88}

The particular solution is:
\[p(x) = p \cos (ax) + q \sin (ax)\]
\item The complimentary function has \(A \cos (ax) + B \sin (ax)\)
\label{sec:orgd36048b}

The particular solution is:
\[p(x) = x(p \cos (ax) + q \sin (ax))\]
\end{enumerate}
\subsubsection{After the particular solution is found}
\label{sec:orgbf8ff0a}
Once we find the particular solution, we must find its first and second derivatives, \(p'(x)\) and \(p''(x)\). After which, we substitute them into the original second-order differential equation to find the constants \(p\) and \(q\). And now, the full general solution to the linear non-homogeneous second-order differential equation is:
\[y = q(x) + p(x)\]
\end{document}
